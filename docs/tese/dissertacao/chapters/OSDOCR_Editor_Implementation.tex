\chapter{OSDOCR Editor - Implementação}
\label{cap_osdocr_editor_implementacao}


% proposito e ferramentas utilizadas

Neste capítulo será analisada a implementação do último módulo da solução OSDOCR, um editor de OCR Tree gráfico. Este vem com o propósito de servir como uma outra aplicação do Toolkit criado, maioritariamente no que toca às ferramentas que rondam a estrutura OCR Tree, assim como um complemento para a OSDOCR Pipeline que, em casos mais complexos, ou para outros tipos de documentos que não foram o foco da solução, requeira uma manipulação dos resultados mais manual. O editor gráfico facilita esta manipulação.

Embora todos os níveis da OCR Tree possam ser transformados com o editor, o objetivo principal é a de trabalhar com os de nível 2, sendo este nível o que será desenhado na tela de edição.

Esta interface gráfica foi criada utilizando para a parte visual e controlo das Views as bibliotecas \textbf{PySimpleGui} e \textbf{MatplotLib}; e para módulo de dados e controlador destes o Toolkit, com a opção de utilizar a Pipeline para a realização de OCR localmente no input.


\section{Sumário}

Segue-se uma listagem das funcionalidades que serão realçadas:

\begin{itemize}\setlength\itemsep{-0.8em}
	
	\item Inputs
	\item Manipulação manual de OCR Tree 
	\item Aplicação local de OCR
	\item Ferramentas disponíveis
		\begin{itemize}\setlength\itemsep{-0.8em}
			\item Divisão de blocos
			\item Categorização de blocos
			\item Ordenação de blocos
			\item Segmentação de blocos em artigos
		\end{itemize}
	\item Outputs
	\item Operações adicionais
\end{itemize}

\section{Funcionalidades}

\subsection{Inputs}

Naturalmente, a primeira funcionalidade a ser discutida são os tipos de ficheiros de entrada necessários e admitidos pelo editor.

Sendo este um editor de OCR Tree, poderia se esperar que o único input necessário seria um ficheiro compatível com OCR Tree, no entanto, a utilidade da ferramenta seria consideravelmente limitada se o utilizador não tivesse a imagem par da OCR Tree para servir como fundo. Tomou-se então a decisão que são necessários 2 inputs para iniciar a tela de edição do GUI, uma imagem de input e um ficheiro convertível para OCR Tree - seja este um Json ou HOCR.

Realça-se que a imagem e a OCR Tree podem não corresponder, sendo responsabilidade do utilizador escolher os inputs corretamente pareados. 

A tela resultará na OCR Tree sobreposta na imagem de input. Se estes estiverem propriamente pareados, os blocos desenhados estarão corretamente alinhados com o texto respetivo na imagem.


%% TODO : ilustracao de editor com inputs escolhidos. talvez apontar para botoes de escolha dos inputs


Numa nota extra sobre os inputs: o editor possui uma opção nas configurações que permite que ao ser escolhida uma imagem de input, seja procurada automaticamente por outputs desta resultantes do uso da pipeline, acelerando o processo de seleção da OCR Tree.


\subsection{Manipulação manual de OCR Tree}

Segue-se o propósito principal da criação do editor gráfico, a facilitação da manipulação manual de OCR Tree.

Utilizando o OSDOCR Editor é possível:

\highlight{Selecionar blocos}[\normalsize] clicando nestes usando o rato. Múltiplos blocos podem ser selecionados e consequentemente manipulados em simultâneo. Para desselecionar um bloco, basta clicar neste novamente.

%% TODO : ilustracao de selecionar blocos: selecionar 1, multiplos, desselecionar


\highlight{Mover blocos}[\normalsize] selecionando um ou múltiplos blocos e, com o botão do rato pressionado, mover o cursor para o local desejado.

%% TODO : ilustracao de mover blocos selecionados


\highlight{Redimensionar blocos}[\normalsize] selecionando um ou múltiplos blocos e, clicando no vértice para pivô de redimensionamento, mover conforme desejado.

%% TODO : ilustracao de redimensionar blocos

\highlight{Atualizar texto do bloco}[\normalsize] manualmente, ao selecionar um bloco e, na aba da direita com a informação do bloco, atualizar o texto de acordo e clicar em salvar. No caso de múltiplos blocos serem selecionados, apenas o último selecionado é modificável.

%% TODO : ilustracao de atualizar texto do bloco

\highlight{Adicionar um novo bloco}[\normalsize] clicando no botão do meio do rato no local desejado, ou utilizando o botão com o mesmo efeito.

%% TODO : ilustracao de criar novo bloco


\highlight{Remover um bloco}[\normalsize] selecionando o bloco a remover e clicando no botão para o efeito. No caso de múltiplos blocos selecionados, apenas o último será removido.

%% TODO : ilustracao de remover um bloco



\subsection{Aplicação local de OCR}

\subsection{Ferramentas disponíveis}

\subsection{Outputs}

\subsection{Operações adicionais}

% funcionalidades disponiveis
%% abir diferentes resultados OCR para uma mesma imagem
%% mover blocos singularmente ou em grupo
%% ajustar dimensoes de blocos
%% OCR de bloco
%%% configuracoes de OCR
%% configuracoes de editor
%% dividir bloco manualmente
%% dividir bloco por espaços vazios
%% categorizar blocos manualmente
%% ordenar blocos
%% calcular ordem de blocos
%% gerar artigos automaticamente
%% gerir artigos
%%% adicionar e remover
%%% adicionar e remover blocos a artigos
%%% alterar ordem dos artigos
%% gerar output
%%% simples
%%% markdown
%%% OCR Tree
%% cache de operacoes
%%% retroceder e reconstruir operacoes





