\chapter{Introdução}

%Contexto, motivação, principais objetivos.

As tecnologias de reconhecimento óptico de caracteres (OCR), tem um papel fundamental na conservação, disponibilização e proliferação de documentos de épocas anteriores à digitalização, ou de origem física sem contrapartida digital. \\
A eficácia desta tecnologia é no entanto dependente de vários fatores: a qualidade das imagens alvo, como a resolução, estado do documento, coloração, qualidade/tipo de escrita; a estrutura dos documentos, quanto mais complexo, mais difícil é obter a informação de forma congruente de forma automática; linguagem do texto, sendo que por vezes diferentes tecnologias, como por exemplo o tesseract, procuram na procura de texto verificar a sua confiança na deteção com o vocabulário conhecido, o qual pode no entanto não coincidir com a época de produção do documento; etc. \\
Estas dependências são especialmente notórios quando se envolvem documentos menos recentes, os quais podem, além de apresentarem envelhecimento causado pelo tempo e danos pelas condições de armazenamento, devido às limitações tecnológicas, assim como por vezes à falta de convenções de formatação dos documentos, não disporem de uma consistência no formato e texto (template, alinhamento, dimensões dos caracteres,etc), usual nos documentos atuais. Estes fatores, resultam então num reconhecimento de texto não tão satisfatórios como se esperaria. \\
Estes documentos, são mais comumente, mais não exclusivamente, reconhecidos como anteriores à era da digitalização, sendo que o foco de trabalho será maioritaremente dirigido a documentos desta época, como jornais, revistas e outros, do século passado ou anteriores. \\
O seguinte documento pretende então servir como um estudo dos desafios apresentados por estes tipos de documentos perante OCR, assim como a procura de soluções para a melhoria dos resultados na deteção de texto e assim criar uma ferramenta que torne o processo de extração de informação destes tipos de documentos mais simples e confiável. \\
O trabalho realizado seguirá então por um processo de investigação do estado da arte, onde serão aprofundados teoricamente e na prática diferentes motores de OCR, de modo a permitir entender a sua capacidade em tratar deste tipo de documentos, e procurar isolar diferentes dificuldades que os documentos lhes apresentam; seguido da utilização do estudo realizado para a criação de uma ferramenta ou conjunto de ferramentas que, fazendo uso destas tecnologias e mitigando ou resolvendo os problemas que elas demonstram na situação descrita, melhore o processo de extração de informação. Esta componente prática, entende então que torne possível a extração de informação dos documentos, isolando artigos ou outras peças contínuas de texto; detetar problemas nas imagens dos documento e aplicar/sugerir diferentes soluções para os corrigir; possibilitar uma modernização automática do texto detetado; habilitar a conversão do documento original (em pdf ou imagem) para um novo tipo de ficheiro com a informação extraída para text (por exemplo: html que mantenha a mesma estrutura do ficheiro original; markdown com os diferentes artigos extraídos isolados; pdf com camadas de texto sobre a imagem, limpo de blocos dispensáveis, com a sua performance de navegação melhorada).