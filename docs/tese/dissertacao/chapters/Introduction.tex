\chapter{Introdução}
\label{cap_introducao}

Neste capítulo, será realizada uma introdução ao problema que o projeto tenciona abordar, composta por uma contextualização do seu estado atual e os desafios que sobre este são impostos. Além disso, os objetivos do trabalho serão listados e será descrita a estrutura do documento.

%Contexto, motivação, principais objetivos.
\section{Enquadramento e motivação}


A digitalização tem um papel fundamental na conservação, disponibilização e proliferação de documentos físicos, não só contemporâneos, mas também de eras anteriores à revolução da informação. Esta tecnologia, acoplada a ferramentas de \acrshort{ocr}, veio trazer uma facilidade de navegação, consulta e manipulação destes documentos que anteriormente não era possível.

A eficácia de \acrshort{ocr} é no entanto dependente de vários fatores nas imagens ou ficheiros alvo: a qualidade das imagens, como a resolução, estado do documento, coloração, qualidade/tipo de escrita; a estrutura dos documentos - quanto mais complexo, mais difícil é obter a informação de forma automática mantendo 
 a congruência original -; linguagem do texto, sendo que por vezes diferentes tecnologias, como por exemplo \textbf{Tesseract}, procuram verificar a sua confiança na deteção com o vocabulário conhecido, o qual pode não coincidir com a época de produção do documento; entre outras. 

Estas dependências são especialmente notórias quando se envolvem documentos mais antigos, os quais podem, além de apresentar envelhecimento causado pelo tempo e danos pelas condições de armazenamento, devido às limitações tecnológicas assim como por vezes à falta de convenções de formatação dos documentos, não dispor de uma consistência no formato e texto (estrutura, alinhamento, dimensões dos caracteres, fonte de texto consistente, etc.) usual nos documentos atuais. Estes fatores resultam então num reconhecimento de texto não tão satisfatórios. 

Estes documentos antigos são mais comummente, mas não exclusivamente, reconhecidos como anteriores à era digital, sendo que o foco de trabalho será maioritariamente dirigido a documentos desta época, como jornais, revistas e outros, do século passado ou anteriores. 
Em especial documentos com estruturas complexas, como é o caso de jornais, onde é possível a segmentação em diferentes partes com conteúdo e propósito distinto e, ao mesmo tempo, uma ordem de leitura complexa i.e., não segue apenas regras simples de posição do conteúdo (texto da esquerda antes do texto da direita e cima antes de baixo), exigindo também noção das características e relação do conteúdo. 

Mesmo para ficheiros do tipo \textbf{hOCR} ou \textbf{PDF}, que já passaram por um processo de reconhecimento de texto, a complexidade da estrutura dos documentos originais ou problemas nos elementos que contém o texto (como por exemplo elementos sobrepostos ou que se intersetam) dificultam a extração e interpretação do seu conteúdo, podendo ser facilmente perdida a lógica original.

Por estas razões, seria útil uma ferramenta que permita uma deteção e tratamento destes documentos de forma automática e de uso simples, permitindo um certo nível de configuração para adaptação entre tipos de documentos com características bem definidas e distintas. 

O presente documento pretende então servir como um estudo dos desafios apresentados por estes tipos de documentos perante \acrshort{ocr}, assim como a procura de soluções para a melhoria dos resultados na deteção  e extração de texto e assim criar uma ferramenta que torne o processo de extração de informação destes tipos de documentos mais simples e fiável. 

Como trabalho complementar, é proposta a implementação de um método de modernização do conteúdo extraído, envolvendo a criação de uma ferramenta capaz de criar dicionários entre diferentes iterações de uma mesma linguagem. 

\section{Objetivos}
\label{section_objetivos}

O principal objetivo deste trabalho é a realização de um estudo sobre os problemas apresentados à extração de conteúdo de documentos de estrutura complexa - mantendo 
a sua lógica original -, assim como a implementação de uma solução para resolver ou mitigar estes desafios, aumentando a confiança na informação extraída. 
Em termos dos casos alvo do trabalho, será prioridade o estudo de jornais com texto máquina. Tal deve-se ao facto de jornais serem um particular tipo de documento que apresenta mais dificuldades e se encontra em maior procura de soluções e, texto máquina por ser mais comum para este tipo de documento. Esta segunda restrição é menos relevante pois não é uma dificuldade do trabalho e pode ser resolvida perante a escolha da tecnologia de reconhecimento utilizada.

Especificando, os objetivos do trabalho são:
\begin{itemize}
    \item Estudar os diferentes softwares de \acrshort{ocr} disponíveis e as diferenças entre estes.
    \item Estudar as dificuldades que documentos podem apresentar no processo de reconhecimento de texto.
    \item Estudar o trabalho desenvolvido sobre a área de tratamento de imagem, identificação de tipo de documento, segmentação de documentos, algoritmos de cálculo da ordem de leitura, melhoria de resultados de \acrshort{ocr} e métricas de validação de resultado \acrshort{ocr}.
    \item Estudar trabalhos com âmbito similar ou relacionado ao presente.
    \item Implementação de um conjunto de ferramentas dirigidas à solução dos problemas propostos.
    \item Implementação de uma ferramenta em formato \acrshort{gui} e comando de consola que aplique uma pipeline cujo input seria um ficheiro - imagem, pdf, hOCR -, identifique e trate de problemas deste se necessário para melhorar os resultados de \acrshort{ocr} e, por fim, devolva um output que mantenha a lógica e conteúdo do documento original.
    \item Secundário : ferramenta para criação de dicionário de diferentes versões de uma linguagem para: modernização de texto; léxico de motor \acrshort{ocr}. Ferramenta tem como input duas versões de um documento na mesma linguagem mas iterações diferentes e dá como output um dicionário entre as versões.
    \begin{itemize}
        \item Estudo sobre criação de léxicos e alinhamento de documentos.
    \end{itemize}
\end{itemize}



\section{Estrutura da dissertação}

Esta dissertação segue a seguinte estrutura:

\begin{itemize}
    \item Capítulo \ref{cap_introducao}: Breve contextualização sobre o tema proposto, as dificuldades impostas por documentos estruturados e com digitalizações ou condições físicas degradadas, nos resultados \acrshort{ocr}, e a utilidade de uma ferramenta para o tratamento destas. Além disso foram listados os objetivos do trabalho.

    \item Capítulo \ref{cap_estado_arte}: Estudo sobre o estado da arte nos tópicos relacionados ao tema da dissertação, as suas dificuldades e soluções destas; estudo de trabalho anteriormente realizado com âmbito similar ao atual ou técnicas relevantes para a construção da solução do problema.

    \item Capítulo \ref{cap_problema}: Listagem dos diferentes problemas que a solução irá abranger e os desafios que estes apresentam. Apresentação do desenho da solução.

    \item Capítulo \ref{cap_contribuicao}: Descrição da solução e ferramentas implementadas.

    \item Capítulo \ref{cap_aplicacoes}: Apresentação e estudo dos resultados do trabalho realizado.

    \item Capítulo \ref{cap_conclusao}: Reflexão sobre o trabalho realizado, os resultados e a experiência obtida, assim como uma breve exploração de caminhos para trabalho futuro do projeto. 

    \item Capítulo \ref{cap_planeamento}: No último capítulo é explicado o plano de desenvolvimento da dissertação.
\end{itemize}
