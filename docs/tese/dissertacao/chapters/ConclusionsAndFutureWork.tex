\chapter{Conclusões e trabalho futuro}
\label{cap_conclusao}

Neste capítulo será feito um sumário do trabalho e estudo realizado e uma introspeção sobre o trabalho futuro.

\section{Conclusões}

O projeto atual, propõe a concretização de uma ferramenta para melhorar os resultados de softwares de reconhecimento de caracteres em documentos estruturados antigos, em particular, jornais. Para isto, nesta primeira fase, foram definidos os objetivos principais do trabalho, assim como algumas vias de expansão consoante o desenrolar da sua implementação. Além disso, foi realizado um estudo sobre o estado da arte com base em dois aspetos principais: softwares \acrshort{ocr} e práticas comuns na sua utilização; e exploração sobre trabalhos relacionados a este tema ou técnicas relevantes para a proposta. Com isto, foi possível entender os desafios mais relevantes que se apresentam ao reconhecimento de caracteres, assim como os procedimentos \textit{standard} para os abordar, nomeadamente: pré processamento de imagem, pós processamento de texto, segmentação e métricas de validação; e algumas soluções focadas em tarefas similares ao do atual trabalho. Por último, realizou-se um compilado de algumas tarefas de implementação já realizadas que auxiliaram na perceção dos desafios impostos no tema e ao mesmo tempo uma melhor perceção sobre o funcionamento e capacidade da tecnologia \acrshort{ocr}.

\section{Perspetiva de trabalho futuro}

Partindo do estado atual do projeto, onde uma base de conhecimento do tema já foi concebida, os futuros passos seguirão maioritariamente na componente prática proposta, i.e. a construção das ferramentas para extração de conteúdo de jornais. Como mencionado nos objetivos, abre-se ainda a possibilidade para um aprofundamento na área de criação de léxicos entre versões de uma mesma linguagem para possibilitar a modernização do conteúdo extraído pela ferramenta principal. 
		