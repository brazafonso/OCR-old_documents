\chapter*{Resumo}
\vspace{-2em}

\newacronym{ocr}{OCR}{reconhecimento óptico de caracteres}
\newacronym{gui}{GUI}{graphic user interface}

A digitalização de documentos permitiu uma nova forma de salvaguardar informação para a posteridade, evitando a sua perda pelo deterioramento físico destes. De forma a posteriormente transcrever estes documentos, permitindo uma consulta, processamento e manipulação mais simples, o uso de software de \acrshort{ocr} é essencial. Esta tecnologia é, no entanto, dependente em diferentes níveis das características do seu alvo, nomeadamente: qualidade da imagem, complexidade da estrutura do documento, linguagem do texto. 

Documentos mais antigos, em especial jornais por possuírem estruturas complexas, apresentam por este motivo resultados que diferem bastante do seu conteúdo original; tanto a nível do texto reconhecido, como da sua organização para os diferentes outputs disponíveis (ex.: txt simples). A tarefa de extrair informação destes documentos, como o isolamento e extração de artigos, torna-se assim complexa e propensa a erros. 

Este trabalho propõe então a criação de uma solução, focada neste tipo de documentos, que afetando os dados que rodeiam o processo de \acrshort{ocr}, i.e. imagem, resultados de \acrshort{ocr} e texto; permita resolver problemas que este apresente, particularmente na dimensão dos resultados de \acrshort{ocr}, e assim proporcione formas de melhorar a transcrição final. 

Na base da solução, está o modelo \textbf{OCR Tree} que procura servir como representação universal dos resultados de OCR ou, por outras palavras, serve como árvore de estrutura e conteúdo documental.

A solução é composta por 3 componentes principais:
\begin{itemize}\setlength\itemsep{-0.5em}
	\vspace{-0.5em}
	\item \textbf{Toolkit}: com ferramentas para análise e tratamento de imagem; processamento de texto; e especial foco na análise e manipulação de OCR Tree.
	\item \textbf{Pipeline}: aplicação do Toolkit e ferramentas auxiliares numa linha de uso de \acrshort{ocr}, configurável. Permitirá obter resultados objetivos sobre o Toolkit.
	\item \textbf{Editor Gráfico de OCR Tree}: interface gráfica que permita a manipulação manual facilitada de OCR Tree. Faz uso do Toolkit e da Pipeline nas operações que disponibiliza.
\end{itemize}




\paragraph{Palavras-chave} OCR, Digitalização, Documentos estruturados, Documentos antigos, Segmentação de documentos, Tratamento de imagem, Editor gráfico

\cleardoublepage

\chapter*{Abstract}

The digitization of documents has opened a new way of preserving information for posterity, avoiding its loss through their physical decay. To allow the transcription of these documents, enabling an easier search, indexation and manipulation of them, the use of \acrshort{ocr} software is essential. This technology is, however, dependent in many ways on the characteristics of its target, namely: the quality of the image, the complexity of the document's structure, the text's language. 

Older documents, especially newspapers for having complex structures, result in poor transcriptions that differ from their original content, both in the recognized text, and in the organization of the available final outputs (ex.: simple txt).
Extracting information from these documents, such as the segmentation and extraction of articles, becomes thus a complex and error prone task. 

This work proposes  a solution, focused on these documents, which by targeting the data that encompasses  the process of \acrshort{ocr}, i.e. image, \acrshort{ocr} results and text; can solve problems that arise in this process, particularly on the issue of \acrshort{ocr} results, thus providing ways to enhance the final transcript.


The basis of this solution is the OCR Tree module, which aims to serve as an universal representation of \acrshort{ocr} results or, in other words, serves represents the structural and content tree of a document.

The solution is composed by 3 main components:
\begin{itemize}\setlength\itemsep{-0.5em}
	\vspace{-0.5em}
	\item \textbf{Toolkit}: with tools for analysis and image treatment; text processing; and special focus on the analysis and manipulation of OCR Tree.
	\item \textbf{Pipeline}: an application  of the Toolkit and auxiliary tools in the form of a configurable pipeline surrounding the use of OCR. Will as a result allow gathering objective  metrics of the Toolkit.
	\item \textbf{OCR Tree Graphical Editor}: graphical interface that allows for an easy manual manipulation of the OCR Tree. Makes use of the Toolkit and the Pipeline in the provided functionalities.
\end{itemize}


\paragraph{Keywords} OCR, Digitalization, Structured documents, Old documents, Document segmentation, Image processing, Graphical Editor

\cleardoublepage
