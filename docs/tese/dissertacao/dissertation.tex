\documentclass[12pt,a4paper]{report}
% \usepackage{dissertation}

%%%%%%%%%%%%%%%%%%%%%%%%%%%% copiado de disseration.sty para funcionar melhor no TeXstudio (apagar quando finalizado)
\usepackage[english, portuguese]{babel}
\usepackage{fontspec}
\usepackage{geometry}
\geometry{top=0mm,bottom=10mm,right=0mm,left=79mm} % for the title page ; new geometry after
\usepackage{subcaption}
\usepackage{adjustbox}
\usepackage{titling}
\usepackage[hidelinks]{hyperref}
\usepackage{natbib}
\usepackage{afterpage}
\usepackage{datetime}
\usepackage{csquotes}
\usepackage{amsmath}
\usepackage{amsthm}
\usepackage{amssymb}
\usepackage{stmaryrd}
\usepackage{mathtools}
\usepackage{listings}
\usepackage{titlesec}
\usepackage{multirow}
\usepackage{epigraph}
\usepackage{datetime}
\usepackage{caption}
\usepackage[many]{tcolorbox} 
\usepackage{tabularx}
\usepackage{leading}
\usepackage[capitalise]{cleveref} % Create better looking section cross-reference links
\usepackage[acronym]{glossaries}
\renewcommand*{\glstextformat}{\textbf}
\usepackage{float}
\usepackage{wrapfig}
\usepackage{enumitem}

%-- Space between lines
\renewcommand{\baselinestretch}{1.5} 

%-- Font News Got T
\setmainfont{NewsGotT.ttf}
\setmainfont[
Ligatures=TeX,
AutoFakeSlant=0.20,
BoldFont=NewsGotTBold.ttf
]{NewsGotT.ttf}

\newfontfamily\NewsGotTLight{NewsGotTLight.ttf}

%-- Cover
\newcommand{\thelogo}{}
\newcommand{\logo}[3]{\renewcommand{\thelogo}{
		\includegraphics[width=26mm]{images/logos/UM.jpg}\includegraphics[width=26mm]{images/logos/#1.jpg}
		
		\vspace{7.05mm}
		\leading{16.8pt}
		{\large
			\textbf{Universidade do Minho}
			\\
			{\NewsGotTLight
				#2
				\\
				#3
		}}
		
		\vspace{37.32mm} %43.24mm - 5.92mm
}}

\newcommand{\thelogoB}{}
\newcommand{\logoB}[3]{\renewcommand{\thelogoB}{
		\includegraphics[width=26mm]{images/logosB/UM.jpg}\hspace{0.5mm}\includegraphics[width=26mm]{images/logosB/#1.jpg}
		
		\vspace{7.05mm}
		\leading{16.8pt}
		{\large
			\textbf{Universidade do Minho}
			\\
			{\NewsGotTLight
				#2
				\\
				#3
		}}
		
		\vspace{37.32mm} %43.24mm - 5.92mm
}}

\newcommand{\thetitleA}{}
\newcommand{\titleA}[1]{\renewcommand{\thetitleA}{#1}}

\newcommand{\thetitleB}{}
\newcommand{\titleB}[1]{\renewcommand{\thetitleB}{#1}}

\newcommand{\thetitleC}{}
\newcommand{\titleC}[1]{\renewcommand{\thetitleC}{#1}}

\newcommand{\themasters}{}
\newcommand{\masters}[1]{\renewcommand{\themasters}{#1 \\}}

\newcommand{\thearea}{}
\newcommand{\area}[1]{\renewcommand{\thearea}{#1 \\}}

\newcommand{\thesupervisor}{}
\newcommand{\supervisor}[1]{\renewcommand{\thesupervisor}{#1}}

\newcommand{\thecosupervisor}{}
\newcommand{\cosupervisor}[1]{\renewcommand{\thecosupervisor}{\textbf{#1}}}

%-- Figures and Tables
\usepackage{chngcntr}
\counterwithout{figure}{chapter}
\counterwithout{table}{chapter}

%-- Date
\def\myear{\ifcase\month\or
	janeiro\or fevereiro\or março\or abril\or maio\or junho\or
	julho\or agosto\or setembro\or outubro\or novembro\or dezembro\fi
	\space\number\year}

%--Colors 
\usepackage{color}
\usepackage{pagecolor}

\definecolor{PANTONECoolGray7C}{HTML}{97999B}
\definecolor{PANTONE1807C}{HTML}{A4343A}
\definecolor{blueUM}{HTML}{24A2DA}

%-- Hypersetup
\hypersetup{
	colorlinks = true,
	linkcolor = blueUM,
	citecolor = blueUM,
	urlcolor = blueUM
}

%-- Title format
\titleformat{\part}[display]
{\relax}
{\centering\huge\bfseries\partname~\thepart}
{0ex}
{\centering\Huge\vspace{0ex}\bfseries}

%\titleformat{\chapter}
%  {\LARGE\bfseries}{\thechapter}{1em}{}

\titleformat{\chapter}[display]
{\relax}
{\Large\bfseries\chaptername~\thechapter}
{0ex}
{\LARGE\vspace{0ex}\bfseries}

\titleformat{\section}
{\Large\bfseries}{\thesection}{1em}{}

\titleformat{\subsection}
{\large\bfseries}{\thesubsection}{1em}{}

%-- Footnote format
\ifdefined\deffootnote
\deffootnote{.5em}{0em}{\thefootnotemark
}
\else
\PassOptionsToPackage{marginal}{footmisc}
\RequirePackage{footmisc}
\fi
\renewcommand{\footnotelayout}{\scriptsize}

%-- Environments
\newenvironment{backcover}{\begin{titlepage}}{\end{titlepage}}

%-- Indexes
\usepackage{imakeidx}

\def\Latex{
	\href{http://www.latex-project.org/}{\LaTeX}
	\index{\LaTeX}}
\newcommand{\tug}[1]{
	\href{http://tug.org/}{#1}
	\index{TeX!TeX Users Group (TUG)}}
\newcommand{\TUG}{
	\tug{TUG}}
\newcommand{\uminho}[1]{
	\href{http://www.uminho.pt}{#1}
	\index{UM!Universidade do Minho}}




%%%%%%%%%%%%%%%%%%%% fim de copia


% Define the highlight command
\usepackage{xparse} % This package is needed for \NewDocumentCommand

\NewDocumentCommand{\highlight}{m O{\large}}{%
	\vspace{1em} % Adds vertical space before the highlighted text
	\par\noindent % Ensures the text starts on a new line and is not indented
	{\textbf{#2 #1}} % Makes the text bold and applies the given text size
	\vspace{1em} % Adds vertical space after the highlighted text
}

\setcounter{tocdepth}{3}
\usepackage{algorithm,algpseudocode,float}
\makeatletter
% \renewcommand{\@algocf@capt@plain}{above}% formerly {bottom}
\makeatother


\makeatletter
\newenvironment{breakablealgorithm}
{% \begin{breakablealgorithm}
		\begin{center}
			\refstepcounter{algorithm}% New algorithm
			\hrule height.8pt depth0pt \kern2pt% \@fs@pre for \@fs@ruled
			\renewcommand{\caption}[2][\relax]{% Make a new \caption
				{\raggedright\textbf{\fname@algorithm~\thealgorithm} ##2\par}%
				\ifx\relax##1\relax % #1 is \relax
				\addcontentsline{loa}{algorithm}{\protect\numberline{\thealgorithm}##2}%
				\else % #1 is not \relax
				\addcontentsline{loa}{algorithm}{\protect\numberline{\thealgorithm}##1}%
				\fi
				\kern2pt\hrule\kern2pt
			}
		}{% \end{breakablealgorithm}
		\kern2pt\hrule\relax% \@fs@post for \@fs@ruled
	\end{center}
}
\makeatother



% \usepackage{algpseudocode}
\makeglossaries
\makeindex


\logo{EE}{Escola de Engenharia}{}
\logoB{EE}{Escola de Engenharia}{}


\author{Gonçalo Braz Afonso}

\titleA{OCR para documentos estruturados antigos}
\titleB{Old structured documents OCR}

\masters{Mestrado em Engenharia Informática}
%\area{Área de especialização}
\supervisor{José João Antunes Guimarães Dias Almeida}




\begin{document}
\setlength{\parindent}{0em}

%-- Covers
\begin{titlepage}
\color{PANTONECoolGray7C}
\thelogo
\leading{20.4pt}
{\Large
\theauthor
\\
%
\\
\textbf{\thetitleA}
\\
\textbf{\thetitleB}
\\
\textbf{\thetitleC}
}

\vspace*{\fill}
{\footnotesize \myear}
\end{titlepage}

\null
\thispagestyle{empty}
\pagecolor{PANTONECoolGray7C}
\afterpage{\nopagecolor}
\newpage

\begin{titlepage}
\color{PANTONECoolGray7C}
\thelogoB
\leading{20.4pt}
{\Large
\theauthor
\\
%
\\
\textbf{\thetitleA}
\\
\textbf{\thetitleB}
\\
\textbf{\thetitleC}
}

\vspace{55.2mm}
\leading{16.8pt}
{\large
Dissertação de Mestrado
\\
\themasters
\\
\thearea
Trabalho efetuado sob a orientação de
\\
\textbf{\thesupervisor}
\\
\thecosupervisor}

\vspace*{\fill}
{\footnotesize \myear}
\end{titlepage}

%-- Document setup
\newgeometry{right=25mm, left=25mm, top=25mm, bottom=25mm}
\pagenumbering{roman}

\setlength{\parskip}{0pt}
\setlength{\parindent}{1.5em}

%-- Preamble
\chapter*{Direitos de Autor e Condições de Utilização do Trabalho por Terceiros}
\setlength{\parskip}{1em}
\noindent
Este é um trabalho académico que pode ser utilizado por terceiros desde que respeitadas as regras e boas práticas internacionalmente aceites, no que concerne aos direitos de autor e direitos conexos.

\noindent
Assim, o presente trabalho pode ser utilizado nos termos previstos na licença abaixo indicada.

\noindent
Caso o utilizador necessite de permissão para poder fazer um uso do trabalho em condições não previstas no licenciamento indicado, deverá contactar o autor, através do RepositóriUM da Universidade do Minho.

\section*{Licença concedida aos utilizadores deste trabalho:}

\textit{[Caso o autor pretenda usar uma das licenças Creative Commons, deve escolher e deixar apenas um dos seguintes ícones e respetivo lettering e URL, eliminando o texto em itálico que se lhe segue. Contudo, é possível optar por outro tipo de licença, devendo, nesse caso, ser incluída a informação necessária adaptando devidamente esta minuta]}

\noindent
\includegraphics[]{images/CCBY.png}
\\
\textbf{CC BY}
\\
\url{https://creativecommons.org/licenses/by/4.0/}
\textit{[Esta licença permite que outros distribuam, remixem, adaptem e criem a partir do seu trabalho, mesmo para fins comerciais, desde que lhe atribuam o devido crédito pela criação original. É a licença mais flexível de todas as licenças disponíveis. É recomendada para maximizar a disseminação e uso dos materiais licenciados.]}

%--

\noindent
\includegraphics[]{images/CCBYSA.png}
\\
\textbf{CC BY-SA}
\\
\url{https://creativecommons.org/licenses/by-sa/4.0/}
\textit{[Esta licença permite que outros remisturem, adaptem e criem a partir do seu trabalho, mesmo para fins comerciais, desde que lhe atribuam o devido crédito e que licenciem as novas criações ao abrigo de termos idênticos. Esta licença costuma ser comparada com as licenças de software livre e de código aberto «copyleft». Todos os trabalhos novos baseados no seu terão a mesma licença, portanto quaisquer trabalhos derivados também permitirão o uso comercial. Esta é a licença usada pela Wikipédia e é recomendada para materiais que seriam beneficiados com a incorporação de conteúdos da Wikipédia e de outros projetos com licenciamento semelhante.]}

%--

\noindent
\includegraphics[]{images/CCBYND.png}
\\
\textbf{CC BY-ND}
\\
\url{https://creativecommons.org/licenses/by-nd/4.0/}
\textit{[Esta licença permite que outras pessoas usem o seu trabalho para qualquer fim, incluindo para fins comerciais. Contudo, o trabalho, na forma adaptada, não poderá ser partilhado com outras pessoas e têm que lhe ser atribuídos os devidos créditos.]}

%--

\noindent
\includegraphics[]{images/CCBYNC.png}
\\
\textbf{CC BY-NC}
\\
\url{https://creativecommons.org/licenses/by-nc/4.0/}
\textit{[Esta licença permite que outros remisturem, adaptem e criem a partir do seu trabalho para fins não comerciais, e embora os novos trabalhos tenham de lhe atribuir o devido crédito e não possam ser usados para fins comerciais, eles não têm de licenciar esses trabalhos derivados ao abrigo dos mesmos termos.]}

%--

\noindent
\includegraphics[]{images/CCBYNCSA.png}
\\
\textbf{CC BY-NC-SA}
\\
\url{https://creativecommons.org/licenses/by-nc-sa/4.0/}
\textit{[Esta licença permite que outros remisturem, adaptem e criem a partir do seu trabalho para fins não comerciais, desde que lhe atribuam a si o devido crédito e que licenciem as novas criações ao abrigo de termos idênticos.]}

%--

\noindent
\includegraphics[]{images/CCBYNCND.png}
\\
\textbf{CC BY-NC-ND}
\\
\url{https://creativecommons.org/licenses/by-nc-nd/4.0/}
\textit{[Esta é a mais restritiva das nossas seis licenças principais, só permitindo que outros façam download dos seus trabalhos e os compartilhem desde que lhe sejam atribuídos a si os devidos créditos, mas sem que possam alterá- los de nenhuma forma ou utilizá-los para fins comerciais.]}

\setlength{\parskip}{0em}
\chapter*{Agradecimentos}
\setlength{\parskip}{1em}

A consolidação do atual trabalho servirá, expectavelmente, como um marco da fase de estudo que, mais ou menos duradoura, a todos marca.

Alcançar este ponto é prova não apenas do meu esforço, mas de todos os que sempre me apoiaram, em especial os meus pais. A educação e formação que me incutiram e os ideais que me constituem são, por base, fruto da sua dedicação e carinho. 

Também a boa disposição e companheirismo inabaláveis da minha irmã não poderiam ser ignorados.

De igual forma agradeço à minha avô, que forneceu os momentos mais ternos da minha infância e juventude, e que sempre teve confiança nas minhas capacidades. 

Agradeço, aos professores e escolas que me formaram durante o caminho e serviram como alicerces da minha aprendizagem. Ao meu orientador, prof. José João por ter sido um guia paciente e elucidador durante este ano de trabalho, pelas inúmeras ideias que sugeriu e tornaram este um trabalho mais interessante, pelas palavras de experiência e as oportunidades que forneceu.

Aos meus amigos que me acompanharam durante este percurso, que forneceram momentos insubstituíveis e laços preciosos, uma saudação calorosa.

A todos os que direta ou indiretamente me marcaram, obrigado!

\setlength{\parskip}{0em}
\chapter*{Declaração de Integridade}
\setlength{\parskip}{1em}
\noindent
Declaro ter atuado com integridade na elaboração do presente trabalho académico e confirmo que não recorri à prática de plágio nem a qualquer forma de utilização indevida ou falsificação de informações ou resultados em nenhuma das etapas conducente à sua elaboração.

\noindent
Mais declaro que conheço e que respeitei o Código de Conduta Ética da Universidade do Minho.


\phantom{space for signature}

\noindent
Universidade do Minho, Braga, \myear

\vspace{25mm}
\noindent\theauthor
\setlength{\parskip}{0em}
\chapter*{Resumo}

\newacronym{ocr}{OCR}{reconhecimento óptico de caracteres}
\newacronym{gui}{GUI}{graphic user interface}

A digitalização de documentos permitiu uma nova forma de salvaguardar informação para a posteridade, evitando a sua perda pelo deterioramento físico destes. De forma a posteriormente transcrever estes documentos, permitindo uma consulta, processamento e manipulação mais simples, o uso de software de \acrshort{ocr} é essencial. Esta tecnologia é, no entanto, dependente em diferentes níveis das características do seu alvo, nomeadamente: qualidade da imagem, complexidade da estrutura do documento, linguagem do texto. 

Documentos mais antigos, em especial jornais por apresentarem estruturas mais complexas, apresentam por este motivo resultados que diferem bastante do seu conteúdo original; tanto a nível do texto reconhecido, como da sua organização para os diferentes outputs disponíveis (ex.: txt simples). A tarefa de extrair informação destes documentos, como por exemplo o isolamento e extração de artigos, torna-se assim complexa e propensa a erros. 

Este trabalho propõe então a criação de uma ferramenta ou um conjunto de ferramentas que permitam auxiliar o processo de extração de conteúdo de documentos, primeiramente mas não exclusivamente, mais antigos e estruturados, com especial foco em jornais. A solução do projeto pretende então ser capaz de detetar e lidar com os diferentes pontos de risco nestes documentos: qualidade da imagem, erros nos resultados de \acrshort{ocr}, segmentação e organização do documento, criação do output organizado. 

Diferentes alternativas para \acrshort{ocr} assim como métodos de tratamento destes problemas serão estudados, comparados, e implementados, de forma a encontrar a melhor solução para a resolução deste problema. O produto final implementado será composto por 3 componentes principais: um Toolkit com ênfase na manipulação de resultados OCR, mas também de tratamento de imagem; uma aplicação do Toolkit na forma de pipeline de OCR, que também explora ferramentas externas; um Editor de resultados OCR, também aplicação do Toolkit, que facilite a manipulação destes devido ao seu aspeto visual, e o teste das outras componentes. 

Na base desta solução, tem o módulo OCR Tree que procura servir como representação universal dos resultados de OCR


\paragraph{Palavras-chave} OCR, Digitalização, Documentos estruturados, Documentos antigos, Segmentação de documentos, Tratamento de imagem, Modernização de texto

\cleardoublepage

\chapter*{Abstract}

The digitization of documents has opened a new way of preserving information for posterity, avoiding its loss through their physical decay. To allow the transcription of these documents, enabling an easier search, indexation and manipulation of them, the use of \acrshort{ocr} software is essential. This technology is, however, dependent in many ways of the characteristics of its target, namely: the quality of the image, the complexity of the document's structure, the text's language. 

Older documents, especially newspapers for having complex structures, result in poor transcriptions that differ from their original content, both in the recognized text, and in the organization of the available final outputs (ex.: simple txt).
Extracting information from these documents, for example, the isolation and extraction of articles, becomes thus a complex and error prone task. 

Therefore, this work aims to create a tool, or a toolkit, that can assist in the process of content extraction from documents, primarily though not exclusively, that are older and structured, specializing in newspapers. The proposed pipeline should then be able to detect and fix potential problems in these documents: image quality, \acrshort{ocr} results errors, segmentation and document organization, restructured output generation.

Different \acrshort{ocr} alternatives, as well as different methods of dealing with these problems, will be studied, compared, and implemented, to find the best solution for the task at hand. The final product will be composed of 3 main components: a Toolkit with emphasis on the manipulation of OCR results, but also image processing; an application of said Toolkit in the shape of an OCR pipeline, which will also explore other external tools; an OCR results Editor, also an application of the Toolkit, which eases the manipulation of these thanks to its visual approach, as well as the testing of the other components.

The basis of this solution is the OCR Tree module, which aims to serve as an universal representation of OCR results.


\paragraph{Keywords} OCR, Digitalization, Structured documents, Old documents, Document segmentation, Image treatment, Text modernization

\cleardoublepage


\phantomsection
\tableofcontents

% Acronyms
\printglossary[type=\acronymtype,nonumberlist, title={Acrónimos}]

% Glossary
\printglossary[title={Glossário}, nonumberlist]

\cleardoublepage
\listoffigures

% List of tables
\listoftables
\clearpage



\cleardoublepage
\pagenumbering{arabic}

%-- Dissertation 

\chapter{Introdução}
\label{cap_introducao}

Neste capítulo, será realizada uma introdução ao problema que o projeto tenciona abordar, composta por uma contextualização do seu estado atual e os desafios que sobre este são impostos. Além disso, os objetivos do trabalho serão listados e será descrita a estrutura do documento.

%Contexto, motivação, principais objetivos.
\section{Enquadramento e motivação}


A digitalização tem um papel fundamental na conservação, disponibilização e proliferação de documentos físicos, não só contemporâneos, mas também de eras anteriores à revolução da informação. Esta tecnologia, acoplada a ferramentas de \acrshort{ocr}, veio trazer uma facilidade de navegação, consulta e manipulação destes documentos que anteriormente não era possível.

A eficácia de \acrshort{ocr} é no entanto dependente de vários fatores nas imagens ou ficheiros alvo: a qualidade das imagens, como a resolução, estado do documento, coloração, qualidade/tipo de escrita; a estrutura dos documentos - quanto mais complexo, mais difícil é obter a informação de forma automática mantendo 
 a congruência original -; linguagem do texto, sendo que por vezes diferentes tecnologias, como por exemplo \textbf{Tesseract}, procuram verificar a sua confiança na deteção com o vocabulário conhecido, o qual pode não coincidir com a época de produção do documento; entre outras. 

Estas dependências são especialmente notórias quando se envolvem documentos mais antigos, os quais podem, além de apresentar envelhecimento causado pelo tempo e danos pelas condições de armazenamento, devido às limitações tecnológicas assim como por vezes à falta de convenções de formatação dos documentos, não dispor de uma consistência no formato e texto (estrutura, alinhamento, dimensões dos caracteres, fonte de texto consistente, etc.) usual nos documentos atuais. Estes fatores resultam então num reconhecimento de texto não tão satisfatórios. 

Estes documentos antigos são mais comummente, mas não exclusivamente, reconhecidos como anteriores à era digital, sendo que o foco de trabalho será maioritariamente dirigido a documentos desta época, como jornais, revistas e outros, do século passado ou anteriores. 
Em especial documentos com estruturas complexas, como é o caso de jornais, onde é possível a segmentação em diferentes partes com conteúdo e propósito distinto e, ao mesmo tempo, uma ordem de leitura complexa i.e., não segue apenas regras simples de posição do conteúdo (texto da esquerda antes do texto da direita e cima antes de baixo), exigindo também noção das características e relação do conteúdo. 

Mesmo para ficheiros do tipo \textbf{hOCR} ou \textbf{PDF}, que já passaram por um processo de reconhecimento de texto, a complexidade da estrutura dos documentos originais ou problemas nos elementos que contém o texto (como por exemplo elementos sobrepostos ou que se intersetam) dificultam a extração e interpretação do seu conteúdo, podendo ser facilmente perdida a lógica original.

Por estas razões, seria útil uma ferramenta que permita uma deteção e tratamento destes documentos de forma automática e de uso simples, permitindo um certo nível de configuração para adaptação entre tipos de documentos com características bem definidas e distintas. 

O presente documento pretende então servir como um estudo dos desafios apresentados por estes tipos de documentos perante \acrshort{ocr}, assim como a procura de soluções para a melhoria dos resultados na deteção  e extração de texto e assim criar uma ferramenta que torne o processo de extração de informação destes tipos de documentos mais simples e fiável. 

Como trabalho complementar, é proposta a implementação de um método de modernização do conteúdo extraído, envolvendo a criação de uma ferramenta capaz de criar dicionários entre diferentes iterações de uma mesma linguagem. 

\section{Objetivos}
\label{section_objetivos}

O principal objetivo deste trabalho é a realização de um estudo sobre os problemas apresentados à extração de conteúdo de documentos de estrutura complexa - mantendo 
a sua lógica original -, assim como a implementação de uma solução para resolver ou mitigar estes desafios, aumentando a confiança na informação extraída. 
Em termos dos casos alvo do trabalho, será prioridade o estudo de jornais com texto máquina. Tal deve-se ao facto de jornais serem um particular tipo de documento que apresenta mais dificuldades e se encontra em maior procura de soluções e, texto máquina por ser mais comum para este tipo de documento. Esta segunda restrição é menos relevante pois não é uma dificuldade do trabalho e pode ser resolvida perante a escolha da tecnologia de reconhecimento utilizada.

Especificando, os objetivos do trabalho são:
\begin{itemize}
    \item Estudar os diferentes softwares de \acrshort{ocr} disponíveis e as diferenças entre estes.
    \item Estudar as dificuldades que documentos podem apresentar no processo de reconhecimento de texto.
    \item Estudar o trabalho desenvolvido sobre a área de tratamento de imagem, identificação de tipo de documento, segmentação de documentos, algoritmos de cálculo da ordem de leitura, melhoria de resultados de \acrshort{ocr} e métricas de validação de resultado \acrshort{ocr}.
    \item Estudar trabalhos com âmbito similar ou relacionado ao presente.
    \item Implementação de um conjunto de ferramentas dirigidas à solução dos problemas propostos.
    \item Implementação de uma ferramenta em formato \acrshort{gui} e comando de consola que aplique uma pipeline cujo input seria um ficheiro - imagem, pdf, hOCR -, identifique e trate de problemas deste se necessário para melhorar os resultados de \acrshort{ocr} e, por fim, devolva um output que mantenha a lógica e conteúdo do documento original.
    \item Secundário : ferramenta para criação de dicionário de diferentes versões de uma linguagem para: modernização de texto; léxico de motor \acrshort{ocr}. Ferramenta tem como input duas versões de um documento na mesma linguagem mas iterações diferentes e dá como output um dicionário entre as versões.
    \begin{itemize}
        \item Estudo sobre criação de léxicos e alinhamento de documentos.
    \end{itemize}
\end{itemize}



\section{Estrutura da dissertação}

Esta dissertação segue a seguinte estrutura:

\begin{itemize}
    \item Capítulo \ref{cap_introducao}: Breve contextualização sobre o tema proposto, as dificuldades impostas por documentos estruturados e com digitalizações ou condições físicas degradadas, nos resultados \acrshort{ocr}, e a utilidade de uma ferramenta para o tratamento destas. Além disso foram listados os objetivos do trabalho.

    \item Capítulo \ref{cap_estado_arte}: Estudo sobre o estado da arte nos tópicos relacionados ao tema da dissertação, as suas dificuldades e soluções destas; estudo de trabalho anteriormente realizado com âmbito similar ao atual ou técnicas relevantes para a construção da solução do problema.

    \item Capítulo \ref{cap_problema}: Listagem dos diferentes problemas que a solução irá abranger e os desafios que estes apresentam. Apresentação do desenho da solução.

    \item Capítulo \ref{cap_contribuicao}: Descrição da solução e ferramentas implementadas.

    \item Capítulo \ref{cap_aplicacoes}: Apresentação e estudo dos resultados do trabalho realizado.

    \item Capítulo \ref{cap_conclusao}: Reflexão sobre o trabalho realizado, os resultados e a experiência obtida, assim como uma breve exploração de caminhos para trabalho futuro do projeto. 

    \item Capítulo \ref{cap_planeamento}: No último capítulo é explicado o plano de desenvolvimento da dissertação.
\end{itemize}

\chapter{Estado da arte}

Estado da arte revisto; trabalho relacionado.

\section{Citações}
Exemplo de uma citação: \cite{GRM97}, cf. esta entrada em \texttt{dissertation.bib}.
Outra forma de citar \citep{KeR88}.

\section{Expressões matemáticas}
A equivalência massa-energia é descrita pela famosa equação
\begin{equation}
E=mc^2
\end{equation}
descoberta em 1905 por Albert Einstein.
Em unidades naturais ($c = 1$), a fórmula expressa a identidade
\[
E=m
\]

\section{Notas de rodapé}
Este é um exemplo de uma nota de rodapé\footnote{The quick brown fox jumps over the lazy dog.}.

\section{Acrónimos e Glossário}
\newacronym{gcd}{MDC}{
Máximo Divisor Comum}
\newacronym{lcm}{MMC}{Mínimo múltiplo comum}
\newglossaryentry{maths}
{
    name=matemática,
    description={Matemática é o que os matemáticos fazem}
}
\newglossaryentry{latex}
{
    name=latex,
    description={É uma linguagem especialmente adequada para
documentos científicos}
}
\newglossaryentry{formula}
{
    name=fórmula,
    description={Expressão matemática}
}

Dado um conjunto de números, existem métodos elementares para calcular
seu \acrlong{gcd}, que é abreviado como \acrshort{gcd}. Este processo
é semelhante ao usado para o \acrfull{lcm}.

O \Gls{latex} é especialmente adequado
para documentos que incluam \gls{maths}. \Glspl{formula} são corretamente e facilmente renderizados a partir do momento que nos habituamos aos comandos.

\section{Índice}

Neste exemplo, várias palavras-chave\index{palavras-chave} importantes serão usadas pelo que merecem aparecer no Índice\index{Índice}.

Os termos no índice também podem ser aninhados \index{Índice!aninhados}.

Cf. o ficheiro \texttt{dissertation.bib} para ver algumas definições como \uminho{UMinho}.

%%
\chapter{OSDOCR Filsofia}
\label{cap_osdocr_filosofia}

Neste capítulo é realizada uma reflexão sobre o problema e as consequentes propostas que são deste retiradas. Serão descritas decisões e modelos fundamentais da solução, assim como a arquitetura desta. 

\section{O problema}

Na sua essência, o problema abordado pode ser descrito como a aquisição de resultados não satisfatórios na aplicação de OCR sob documentos antigos com texto estruturado. 

Como se analisou no capítulo \ref{cap_estado_arte}, estes resultados podem ser melhorados através de múltiplas interações em múltiplas partes do processo do reconhecimento de texto: quer seja antes deste ser realizado, através de processamento de imagem; durante a deteção de texto, através da configuração proposta ao motor OCR; ou após através do tratamento dos resultados e do texto.

Estas diferentes interações são, na generalidade, aplicações sobre um input inicial de forma a transformá-lo, consequentemente resolvendo particularidades deste. Exemplos disto são: aplicação de técnicas de binarização que reduzem ruído ou imperfeições numa imagem; melhoria de resolução de uma imagem; correções ortográficas de texto; etc. Estas interações, podem então ser consideradas como um conjunto de ferramentas que podem ser aplicadas ao input do nosso problema de forma a abordar diferentes defeitos deste de acordo com as suas características; até porque, como também foi descrito para algumas destas, a sua utilização indiscriminada pode deteriorar, ao invés de melhorar, o produto final. As ferramentas comummente utilizadas não são explicitamente dedicadas ao problema de OCR em concreto, aumentando a sua utilidade e adaptabilidade para outras situações.
Tal reforça o intuito base do trabalho, da criação de um toolkit que proporcione um conjunto de funcionalidades independentes, com a intenção final de procurar melhorar os resultados da transcrição de texto.

Podemos então declarar aqui uma primeira proposta: 

\highlight{Proposta 1: A solução para o problema deve ser composta por um conjunto de ferramentas independentes, i.e. um toolkit.}[\normalsize]

Durante a análise do estado da arte, notou-se que para soluções para o problema de melhorar a aplicação de OCR, um processo com 3 partes principais pode ser identificado, sendo estas: pré-processamento de OCR, composto por processamento do input para realização de OCR, usualmente uma imagem; OCR, configurando o motor de OCR de acordo com as características do input, como a linguagem esperada do texto ou o dpi da imagem; pós-processamento de OCR, aplicando transformações nos resultados como, remover blocos de ruído, ordenar blocos identificados ou aplicar correções no texto. 

\begin{figure}[H]
     \centering
     \includegraphics[width=1\textwidth]{images/diagramas/pipeline_alto_nivel.png}
     \caption{Aplicação de OCR com passos para melhoria de resultados}
     \label{fig:pipeline_high_level}
\end{figure}
 
Observando a figura \ref{fig:pipeline_high_level}, podemos identificar que a estrutura das soluções para aplicação de OCR é uma pipeline.

Estas pipelines podem ser elaboradas como uma sequência de aplicações das ferramentas acima mencionadas, como por exemplo:

\begin{figure}[H]
	\centering
	\includegraphics[width=1\textwidth]{images/diagramas/pipeline_exemplo.png}
	\caption{Exemplo de pipeline de aplicação de OCR}
	\label{fig:pipeline_example}
\end{figure}
 
 
Assumindo então a proposta da criação de um toolkit, e a necessidade de verificar a eficácia deste e criar casos de uso deste, torna-se claro que será útil o desenvolvimento de uma pipeline de aplicação de OCR, que faça uso do toolkit desenvolvido, assim como de outras ferramentas disponíveis e que abordem questões em falta no toolkit.

Seguindo o estudo realizado, esta pipeline será composta por 3 partes principais: pré-OCR, OCR e pós-OCR.


% proposta de pipeline

% proposta de pipeline composta por 3 areas principais, processamento de imagem, OCR, processamento de resultados


\highlight{Proposta 2: A solução para o problema deve possuir uma pipeline de aplicação de OCR que faça uso da toolkit desenvolvida.}[\normalsize]

\highlight{Proposta 3: A pipeline desenvolvida deve ser composta por 3 partes principais: pré-processamento de OCR, OCR e pós-processamento de OCR.}[\normalsize]
 
 
Como discutido, o uso de ferramentas independentes é interessante devido ao facto que diferentes inputs irão necessitar de tratamentos distintos de forma a obter os melhores resultados. Desta forma, é interessante que a pipeline desenvolvida não seja totalmente sequencial, de modo a aumentar a sua utilidade.

% proposta de pipeline composta por blocos opcionais

\highlight{Proposta 4: A pipeline deverá ser configurável. Permitirá, dentro dos blocos disponíveis, escolher aqueles que são aplicados.}[\normalsize]

Esta pipeline, seguindo a estrutura explicada de aplicação de OCR, apresentará 3 partes principais onde, a primeira - pré-OCR - irá lidar com um input do tipo imagem e terá como output uma imagem; a segunda - OCR - terá como input uma imagem e devolverá os resultados de OCR; e a última - pós-OCR - irá ter como input os resultados de OCR, e terá como output a transcrição do texto da imagem.

Nota-se uma ambiguidade no output da 2º parte e no input da 3º, os resultados de OCR. Tal, deve-se ao facto de motores OCR, como o Tesseract, permitirem vários tipos de output. É então relevante a criação de um estrutura de dados universal para a representação de resultados OCR.

Como estudado no capítulo \ref{cap_estado_arte}, tais representações já possuem um standard um formato de ficheiro, como \textbf{HOCR}, portanto a estrutura de dados escolhida deve ser baseado nestes, ou ser convertível para o standard.

% proposta de estrutura de dados unica para a represantacao de resultados OCR

\highlight{Proposta 5: Criação de uma estrutura de dados universal para a representação dos resultados de OCR.}[\normalsize]

\highlight{Proposta 6: A estrutura de dados universal para a representação dos resultados de OCR deve ser baseada ou convertível num formato standard.}[\normalsize]


No estudo do estado da arte, foi possível compreender que a criação de algoritmos ubíquos para todos os tipos de documentos é um empreendimento com diminutas chances de sucesso, sendo que para muitos problemas, como por exemplo o cálculo da ordem de leitura, mesmo focando num só tipo de documento como jornais, os resultados podem não ser satisfatórios devido à diversidade de estruturas existentes. Consequentemente, é esperado que a pipeline não seja sempre suficiente na resolução do problema. 
Deste modo, servindo também como outro caso de uso do toolkit, a criação de uma ferramenta que permite ao utilizador um maior manuseamento dos resultados de OCR é uma proposta para a solução. 

Tendo em conta a forte qualidade visual desta área de trabalho, onde se procura processar e analisar imagens de documentos, a utilidade e usabilidade desta ferramenta de edição é elevada ao definirmos que o editor seja gráfico.

% proposta de editor de estrutura de dados para realizacao de afinações manuais

\highlight{Proposta 7: A solução do problema deverá possuir um editor gráfico dos resultados de OCR, que permita aplicar ferramentas propostas pelo toolkit.}[\normalsize]


% apanhado das proposta

Em suma, as propostas definidas para o desenho da solução são:

\begin{enumerate}[label=\textbf{\arabic*}]\setlength\itemsep{-0.8em}
	\item A solução para o problema deve ser composta por um conjunto de ferramentas independentes, i.e. um toolkit.
	\item A solução para o problema deve possuir uma pipeline de aplicação de OCR que faça uso da toolkit desenvolvida.
	\item A pipeline desenvolvida deve ser composta por 3 partes principais: pré-processamento de OCR, OCR e pós-processamento de OCR.
	\item A pipeline deverá ser configurável. Permitirá, dentro dos blocos disponíveis, escolher aqueles que são aplicados.
	\item Criação de uma estrutura de dados universal para a representação dos resultados de OCR.
	\item A estrutura de dados universal para a representação dos resultados de OCR deve ser baseada ou convertível num formato standard.
	\item A solução do problema deverá possuir um editor gráfico dos resultados de OCR, que permita aplicar ferramentas propostas pelo toolkit.
\end{enumerate}


\section{Modelos}

\section{Arquitetura da solução}
\chapter{OSDOCR Estruturas de Dados - Implementação}
\label{cap_osdocr_estrutura_dados_implementacao}

Neste capítulo, será descrito em detalhe a implementação dos modelos principais de dados usados na base do projeto. Estes são a estrutura OCR Tree, utilizada para representação de resultados OCR; e a estrutura Box, utilizada maioritariamente para a representação de Bounding Boxes, mas com a particularidade de possuir uma coleção de métodos que permitem a sua manipulação.


\section{OCR Tree}
\label{ocr_tree}

Como o produto final do projeto intende aceitar diferentes tipos de resultados OCR, i.e. resultantes de diferentes motores OCR ou de ficheiros como hOCR que já possuem os resultados, existe uma necessidade de converter estes diferentes formatos num único tipo que mantenha a informação base pretendida.

Estruturas de dados standard como \citep{hocr_doc} ou \citep{alto_doc} apresentam um resultado final semelhante e com capacidade base de armazenamento de meta-dados superior porém, sendo baseados em XML, tornam a sua manipulação mais complexa e, em múltiplos casos a informação proporcionada é além do necessário ou gera conclusões erradas quando gerado de output automático (ex.: atribuição de classes caption a blocos que são títulos). Assim sendo, embora tenha sido desenvolvido um conversor de, e para HOCR, para o atual projeto, optou-se pela criação de uma estrutura de dados própria.

Deste modo, tomando como inspiração os atributos dos resultados do Tesseract no modo de dicionário \citep{tesseract_doc}, foi implementada uma estrutura de dados no formato de árvore de dados.

A escolha de uma estrutura de árvore permite a hierarquização de blocos de acordo com o seu nível, quer exista uma divisão de nível à partida, como é o caso do Tesseract que segue: página $\longrightarrow$ bloco $\longrightarrow$ parágrafo $\longrightarrow$ linha $\longrightarrow$ palavra; ou apenas um único nível, semelhante ao Keras-OCR.

Todos os algoritmos desenvolvidos, inclusive os métodos para visualização (métodos de debugging e GUI desenvolvido), assumem e trabalham com os dados de OCR no formato desta estrutura de dados.

As características mais relevantes desta estrutura são:

\begin{itemize}\setlength\itemsep{-0.3em}
	\item \textbf{Level} : Nível/altura do nodo.
	\begin{itemize}\setlength\itemsep{-0.3em}
		\item documento : 0
		\item página 	: 1
		\item bloco		: 2
		\item parágrafo : 3
		\item linha 	: 4
		\item palavra	: 5
	\end{itemize}\setlength\itemsep{-0.3em}
	\item \textbf{(page|block|par|line|word)\_num}: Identificação da ordem (dentro de outras caixas(ex.: linha), se aplicável)
	\item \textbf{text} : Texto do bloco, normalmente apenas preenchido ao nível da palavra
	\item \textbf{conf} : Confiança no texto
	\item \textbf{id}
	\item \textbf{type} : Tipo do bloco, ex.: delimitador, título
	\item \textbf{children}
	\item \textbf{box}: Bounding box do nodo, representado pela estrutura de dados Box, que também possui métodos para transformações e verificações geométricas ou de características.
	\item Características de texto: ex.: texto iniciado (start\_text); texto não terminado (end\_text).
\end{itemize}

%% TODO : Ilustracao de uso de arvore OCR para demonstrar diferentes niveis de um documento

Construtores da classe são capazes de admitir outros atributos não base de modo a expandir a utilidade da estrutura. Construtores disponíveis: iniciação por argumentos, dicionário, ficheiro JSON e ficheiro HOCR.

Da mesma forma, conversores para estes ficheiros compreendidos para iniciação também foram desenvolvidos.

A classe possuí por métodos de transformação e análise sobre a árvore OCR que facilitam a manipulação dos resultados OCR. 

Segue-se uma lista dos métodos mais relevantes disponíveis da classe.

\highlight{id\_boxes}
 
 
\textbf{Descrição:} Adiciona identificador aos blocos.
	
\textbf{Argumentos:}
	\begin{itemize}\setlength\itemsep{-0.3em}
		\item level : lista de níveis onde adicionar identificador
		\item ids (opt): dicionário de ids a utilizar caso não se queira iniciar no 0.
		\item delimiters (opt): flag para identificar delimitadores
		\item area (opt): argumento do tipo Box, que restringe os nodos a identificar a uma dada área
		\item override (opt): flag para reescrever id se já existe.
	\end{itemize}
				
\highlight{calculate\_mean\_height}

\textbf{Descrição:} Calcula a altura média das caixas de um dado nível.
	
\textbf{Argumentos:}
\begin{itemize}\setlength\itemsep{-0.3em}
	\item level : nível a calcular
	\item conf (opt): valor de confiança de texto no caso de apenas serem relevantes caixas com certa confiança (aplicável apenas para nível de texto)
\end{itemize}

	
\highlight{is\_text\_size}

\textbf{Descrição:} Verifica se um nodo se encontra dentro do tamanho de texto.
	
\textbf{Argumentos:}
\begin{itemize}\setlength\itemsep{-0.3em}
	\item text\_size : tamanho de texto a comparar
	\item mean\_height (opt): altura do bloco, caso já tenha sido calculado
	\item range : margem de erro aceitável (relativo)
	\item level : nível das caixas usado caso seja necessário calcular a altura média
	\item conf : confiança do texto a utilizar para calcular a altura média
\end{itemize}

\highlight{is\_empty}

\textbf{Descrição:} Verifica se um nodo é vazio.
	
\textbf{Argumentos:}
\begin{itemize}\setlength\itemsep{-0.3em}
	\item conf : confiança de texto a utilizar para considerar palavras válidas
	\item only\_text : flag que dita se o tipo do bloco influencia o resultado, i.e. blocos de tipo "image" não são vazios
\end{itemize}

	
\highlight{text\_is\_title}

\textbf{Descrição:} Verifica se um nodo é potencial título.
	
\textbf{Algoritmo:} Caixa não é texto vertical e é maior do que o tamanho normal de texto.


\textbf{Argumentos:}
\begin{itemize}\setlength\itemsep{-0.3em}
	\item normal\_text\_size : tamanho de texto considerado como normal
	\item conf : confiança de texto a utilizar para considerar palavras válidas
	\item range : margem de acerto aceitável (relativo)
	\item level : nível usado para calcular o tamanho médio do bloco
\end{itemize}

	
\highlight{is\_delimiter}

\textbf{Descrição:} Verifica se um nodo é potencial delimitador.
	
\textbf{Algoritmo:} Caixa já é do tipo delimitador, ou é vazia e segue a regra:

$ box.width >= box.height*4 || box.height >= box.widht*4 $.


\textbf{Argumentos:}
\begin{itemize}\setlength\itemsep{-0.3em}
	\item conf : confiança de texto a utilizar para considerar palavras válidas
	\item only\_type : flag que dita se usa apenas o tipo do nodo para a verificação
\end{itemize}

	
\highlight{is\_image}

\textbf{Descrição:} Verifica se um nodo é potencial imagem.
	
\textbf{Algoritmo:} Caixa já é do tipo imagem ou, é vazia, não é um delimitador e é 3 vezes mais alta do que o tamanho de texto.


\textbf{Argumentos:}
\begin{itemize}\setlength\itemsep{-0.3em}
	\item conf : confiança de texto a utilizar para considerar palavras válidas
	\item text\_size : tamanho de texto a utilizar para comparação com altura da caixa
	\item only\_type : flag que dita se usa apenas o tipo do nodo para a verificação
\end{itemize}


\highlight{get\_boxes\_in\_area}

\textbf{Descrição:} Obtém todas as caixas numa dada área.


\textbf{Argumentos:}
\begin{itemize}\setlength\itemsep{-0.3em}
	\item area : área de interesse
	\item level : nível dos nodos a ir buscas. Se nível == -1, obtém todos os nodos
	\item conf : confiança de texto a utilizar para considerar nodos válidos
	\item ignore\_type : tipos de nodo a ignorar
\end{itemize}
	
	
\highlight{is\_vertical\_text}

\textbf{Descrição:} Verifica se um nodo é texto vertical.

\textbf{Argumentos:}
\begin{itemize}\setlength\itemsep{-0.3em}
	\item conf : confiança de texto a utilizar para considerar palavras válidas
\end{itemize}
	
\textbf{Algoritmo:}

\begin{breakablealgorithm}
	\caption{Verificação de texto vertical}
	\footnotesize
	\begin{algorithmic}[1]
		\If{nodo não é vazio}
			\State lines
			\If{len(lines) == 0}
				\Return False
			\EndIf
			\State \textit{// Linha única}
			\If{len(lines) == 1}
				\State words
			 \State \textit{// Palavra única}
				\If{len(words) == 1}
					\If{altura da palavra >= 2 * largura da palavra}
						\Return True
					\EndIf
				\State \textit{// Múltiplas palavras} 
				
				\Else
					\State \textit{// Verifica se a maioria das palavras coincidem horizontalmente}
					\State widest\_word <- calcula palavra mais larga
					\State overlapped\_words = 0
					\For{word in words}
						\If{word == widest\_word}
							\State continue
						\EndIf
						\If{word.box.within\_horizontal\_boxes(widest\_word.box,range=0.1)}
							\State overlapped\_words += 1
						\EndIf
					\EndFor
					\If{overlapped\_words/len(words) >= 0.5}
						\Return True
					\EndIf
					
				\EndIf
				
			\State \textit{// Múltiplas linhas} 
			
			\Else
				\State \textit{// Verifica se a maioria das linhas coincidem verticalmente}
				\State tallest\_line <- calcula linha mais alta
				\State overlapped\_lines = 0
				
				\For{line in lines}
					\If{line == tallest\_line}
						\State continue
					\EndIf
					\If{line.box.withinvertical\_boxes(tallest\_line.box,range=0.1)}
						\State overlapped\_lines += 1
					\EndIf
				\EndFor
				\If{overlapped\_lines/len(lines) >= 0.5}
					\Return True
				\EndIf
			
			\EndIf
			
		\EndIf
		
		\Return	False
		
	\end{algorithmic}
\end{breakablealgorithm}



\highlight{prune\_children\_area}

\textbf{Descrição:} Atualiza dimensões dos filhos de um nodo para se encaixarem dentro de uma área.


\textbf{Argumentos:}
\begin{itemize}\setlength\itemsep{-0.3em}
	\item area : área de interesse
\end{itemize}


\highlight{boxes\_below}
(método semelhante para as outras direções)

\textbf{Descrição:} Dada uma lista de OCR Tree, devolve aqueles que se encontram por baixo do bloco atual. Os blocos filtrados podem intersetar ou estar dentro do bloco comparado.


\textbf{Argumentos:}
\begin{itemize}\setlength\itemsep{-0.3em}
	\item blocks : lista de blocos a filtrar
\end{itemize}



\highlight{boxes\_directly\_below}
(método semelhante para as outras direções)

\textbf{Descrição:} Dada uma lista de OCR Tree, devolve aqueles que se encontram diretamente por baixo do bloco atual. Blocos filtrados não estão dentro do bloco comparado e não podem estar diretamente por baixo dos outros blocos.


\textbf{Argumentos:}
\begin{itemize}\setlength\itemsep{-0.3em}
	\item blocks : lista de blocos a filtrar
\end{itemize}
	
	
	
\highlight{join\_trees}

\textbf{Descrição:} Junta duas OCR Tree com do mesmo nível numa única árvore. Tem dois métodos principais de junção das árvores: vertical, operação mais simples em que basicamente apenas se juntam as duas listas de ramos dos nodos raíz (assume-se que uma das árvores é mais alta do que a outra e não se intersetam); e horizontal, onde se procura juntar árvores que têm interseção no eixo y, sendo necessário verificar as posições que os filhos devem tomar e se certos filhos devem ser unidos num único (podendo resultar numa junção de linhas).

%% TODO : ilustração poderá ajudar


\textbf{Argumentos:}
\begin{itemize}\setlength\itemsep{-0.3em}
	\item tree : árvore a juntar
	\item orientation : orientação da junção, vertical ou horizontal.
\end{itemize}

\textbf{Algoritmo:}

\begin{breakablealgorithm}
	\caption{Junção horizontal}
	\footnotesize
	\begin{algorithmic}[1]
		\State tree\_children
		\State self\_children
		\State \textit{// no último nível, filhos são ordenados da esquerda para a direita}
		\If{último nível da tree}
			\State tree\_children $\leftarrow$ ordenar lista da esquerda para a direita
		\EndIf
		
		\For{child in tree\_children}
			\If{não é o último nível}
				\State self\_children $\leftarrow$ ordena de cima para baixo
				\If{child pode ser inserida no topo ou fundo da lista}
					\State self\_children $\leftarrow$ insere child no início ou fim
				\Else
					\State \textit{// procura slot para inserir, ou nodo com quem unir}
					\State joined = False
					\For{i in range(len(self\_children))}
						\If{child não interseta com nodo i ou nodo i+1}
							\State self.children $\leftarrow$ adiciona child entre os dois nodos
							\State joined = True
						\\ElsIf{interseta com nodo i}
							\If{interseção em pelo menos 70\% da altura da caixa}
								\If{nodo i tem filhos}
									\State \textit{// join recursivo}
									\State self\_children[i].join\_trees(child,orientation=orientation)
								\Else
									\State self\_children $\leftarrow$ insere child depois do nodo i
								\EndIf
								\State joined = True
							\Else
								\State \textit{// procura local mais baixo para inserir (por poder intersetar com varios blocos)}
								\For{j in range(i,len(self\_children))}
									\If{nodo j mais alto do que child}
										\State self\_children $\leftarrow$ insere child depois do nodo j
										\State joined = True
									\EndIf
								\EndFor
								\If{not joined}
									\State self\_children $\leftarrow$ insere child no fim
									\State joined = True
								\EndIf
							\EndIf
						\EndIf
						
						\If{joined}
							\State break
						\EndIf
					\EndFor
				\EndIf
			\Else
				\State self.children $\leftarrow$ adiciona child no fim da lista
			\EndIf
			
			\State self\_children $\leftarrow$ atualiza lista de filhos
		\EndFor
		
		
	\end{algorithmic}
\end{breakablealgorithm}


\highlight{remove\_blocks\_inside}

\textbf{Descrição:} Remove os blocos dentro do bloco com dado id. Blocos removidos são do mesmo nível que o bloco com dado id.


\textbf{Argumentos:}
\begin{itemize}\setlength\itemsep{-0.3em}
	\item id : id do bloco a limpar
	\item block\_level : nível do bloco a limpar
\end{itemize}

\highlight{update\_position}

\textbf{Descrição:} Atualiza a posição da bounding box de um nodo e dos seus filhos. Especialmente útil para o editor OCR.


\textbf{Argumentos:}
\begin{itemize}\setlength\itemsep{-0.3em}
	\item top : valor a atualizar verticalmente
	\item left : valor a atualizar horizontalmente
	\item absolute : flag que indica se operação é de do tipo absoluta, i.e. bounding box vai ser diretamente atualizada com estes valores, ou relativa, aos valores da bounding box serão somados os argumentos
\end{itemize}

\highlight{update\_size}

\textbf{Descrição:} Atualiza o tamanho da bounding box de um nodo e dos seus filhos nas arestas (filhos interiores não serão alterados). Especialmente útil para o editor OCR.


\textbf{Argumentos:}
\begin{itemize}\setlength\itemsep{-0.3em}
	\item top : valor a atualizar ao topo
	\item left : valor a atualizar à esquerda
	\item bottom : valor a atualizar ao fundo
	\item right : valor a atualizar à direita
	\item absolute : flag que indica se operação é de do tipo absoluta, i.e. bounding box vai ser diretamente atualizada com estes valores, ou relativa, aos valores da bounding box serão somados os argumentos
\end{itemize}

\highlight{update\_box}

\textbf{Descrição:} Atualiza diretamente valor da bounding box do nodo e dos filhos. Especialmente útil para o editor OCR.

\textbf{Argumentos:}
\begin{itemize}\setlength\itemsep{-0.3em}
	\item top : valor a atualizar ao topo
	\item left : valor a atualizar à esquerda
	\item bottom : valor a atualizar ao fundo
	\item right : valor a atualizar à direita
	\item children : flag que indica se é para se aplicar ajuste direto no nodo, ou apenas ajustar de forma a não sair da bounding box do pai.
\end{itemize}

\highlight{scale\_dimensions}

\textbf{Descrição:} Escala dimensões da bounding box do nodo e dos seus filhos. Especialmente útil para o editor OCR.

\textbf{Argumentos:}
\begin{itemize}\setlength\itemsep{-0.3em}
	\item scale\_width : escalar de valores do eixo horizontal
	\item scale\_height : escalar de valores do eixo vertical
\end{itemize}


\section{Box}
\label{box_data_structure}

A estrutura de dados Box é utilizada maioritariamente para encapsular os dados das bounding boxes dos resultados de OCR. Embora em geral este tipo de dados seja geralmente fornecido por métodos de módulos de manipulação de imagens na forma de tuplo, a utilização de uma classe dedicada permite o desenvolvimento e utilização de métodos para sua manipulação de forma mais simples e organizada.

Tal como a estrutura de dados OCR Tree, esta classe apresenta construtores e conversores de ficheiros diferentes tipos: argumentos simples, dicionário, ficheiro JSON.

Os principais atributos da estrutura são:

\begin{itemize}\setlength\itemsep{-0.3em}
	\item \textbf{left} 	: Valor mais à esquerda da caixa.
	\item \textbf{right}	: Valor mais à direita da caixa.
	\item \textbf{top} 		: Valor mais em cima da caixa (menor do que bottom por ser baseado em manipulação de imagem). 
	\item \textbf{bottom} 	: Valor mais em baixo da caixa.
	\item \textbf{width} 	: Comprimento da caixa.
	\item \textbf{height} 	: Altura da caixa.
\end{itemize}

Realça-se que os atributos desta classe são esperados no formato de inteiros, devido a ter como foco o seu uso no contexto do espaço de imagens.

Segue-se uma lista dos métodos mais relevantes disponíveis da classe.

\highlight{update}

\textbf{Descrição:} Atualiza os valores dos atributos de posição da caixa. Atributos de posição são mantidos válidos, i.e. left <= right e top <= bottom . Altura e comprimento são atualizados automaticamente.

\textbf{Argumentos:}
\begin{itemize}\setlength\itemsep{-0.3em}
	\item top : valor a atualizar ao topo
	\item left : valor a atualizar à esquerda
	\item bottom : valor a atualizar ao fundo
	\item right : valor a atualizar à direita
\end{itemize}


\highlight{move}

\textbf{Descrição:} Soma valores aos atributos de posição da caixa.

\textbf{Argumentos:}
\begin{itemize}\setlength\itemsep{-0.3em}
	\item x : valor a somar nos atributos de posição horizontais
	\item y : valor a somar nos atributos de posição verticais
\end{itemize}

\highlight{within\_vertical\_boxes}
(método semelhante para direção horizontal)

\textbf{Descrição:} Verifica se caixa e caixa a ser comparada estão alinhadas verticalmente, podendo considerar uma margem de acerto. Verificação é realizada nos dois sentidos, i.e. caixa 1 alinhada com caixa 2 ou vice-versa.

\textbf{Argumentos:}
\begin{itemize}\setlength\itemsep{-0.3em}
	\item box : caixa a comparar
	\item range : valor relativo da altura da caixa, a servir como margem para considerar na verificação
\end{itemize}

\highlight{is\_inside\_box}

\textbf{Descrição:} Verifica se caixa a ser comparada está dentro da caixa. Caixa a ser compara tem de estar completamente dentro para resultado afirmativo.

\textbf{Argumentos:}
\begin{itemize}\setlength\itemsep{-0.3em}
	\item box : caixa a comparar
\end{itemize}


\highlight{intersects\_box}

\textbf{Descrição:} Verifica se caixa a ser comparada interseta com a caixa.

\textbf{Argumentos:}
\begin{itemize}\setlength\itemsep{-0.3em}
	\item box : caixa a comparar
	\item extend\_vertical : flag para indicar se verificação deve ser feita apenas longo do eixo x (ex.: utilizado para verificar se caixa comparada está diretamente por acima da caixa)
	\item extend\_horizontal : flag para indicar se verificação deve ser feita apenas longo do eixo y (ex.: utilizado para verificar se caixa comparada está diretamente à direita caixa)
	\item inside : flag para indicar se verificação de caixa dentro conta como interseção
\end{itemize}


\highlight{intersect\_area\_box}

\textbf{Descrição:} Calcula a caixa de interseção entre a caixa e uma caixa a comparar.

\textbf{Argumentos:}
\begin{itemize}\setlength\itemsep{-0.3em}
	\item box : caixa a comparar
\end{itemize}


\highlight{remove\_box\_area}

\textbf{Descrição:} Remove área da caixa. Procura remover área aplicando as menores modificações possíveis. Apenas realiza modificações, se área fornecida está dentro da caixa.

\textbf{Argumentos:}
\begin{itemize}\setlength\itemsep{-0.3em}
	\item area : area da caixa a remover
\end{itemize}


\highlight{get\_box\_orientation}

\textbf{Descrição:} Método naive para obter orientação da caixa (horizontal, vertical ou square) de acordo com a diferença entre a sua altura e comprimento.


\highlight{join}

\textbf{Descrição:} Une duas caixas.

\textbf{Argumentos:}
\begin{itemize}\setlength\itemsep{-0.3em}
	\item box : caixa a unir
\end{itemize}

\highlight{distance\_to}

\textbf{Descrição:} Calcula distância entre duas caixas. Procura dois pontos mais próximos de acordo com argumentos dados e utiliza distância euclidiana para calcular a distância.

\textbf{Argumentos:}
\begin{itemize}\setlength\itemsep{-0.3em}
	\item box : caixa a comparar
	\item border (opt): borda da caixa a ter em conta. Valores disponíveis: "left", "right", "top", "bottom", "closest". Se "closest" for fornecido, procura a menor distância entre bordas. Se nenhum valor for fornecido, utilizado os pontos centrais das caixas.
\end{itemize}


\highlight{distance\_to\_point}

\textbf{Descrição:} Calcula distância entre a caixa e um ponto. Procura calcular a menor distância da caixa ao ponto (tendo em conta a diferença entre o ponto e as bordas).

\textbf{Argumentos:}
\begin{itemize}\setlength\itemsep{-0.3em}
	\item x : valor x do ponto
	\item y : valor y do ponto
\end{itemize}

\highlight{vertices}

\textbf{Descrição:} Retorna uma lista dos vértices da caixa na forma de tuplos (x,y) seguindo de cima para baixo, esquerda para a direita.


\highlight{closest\_edge\_point}

\textbf{Descrição:} Calcula a borda mais próxima entre a caixa e um ponto. Utilizado no editor de resultados OCR para operações de divisão de blocos.

\textbf{Argumentos:}
\begin{itemize}\setlength\itemsep{-0.3em}
	\item x : valor x do ponto
	\item y : valor y do ponto
\end{itemize}







\chapter{OSDOCR Toolkit - Implementação}
\label{cap_osdocr_toolkit_implementacao}

Neste capítulo será discutida a componente de Toolkit, premissa base do tema da dissertação. Esta componente consiste num conjunto de ferramentas focado na melhoria dos resultados obtidos da aplicação de \acrshort{ocr} em documentos antigos, com especial interesse em jornais. 

Estas ferramentas são então pertinentes para os diversos passos do processo convencional de \acrshort{ocr}, i.e. pré-processamento, OCR e pós-processamento; atendendo tanto a processamento de imagem, processamento de resultados de OCR e texto, e validação de resultados.


\section{Sumário}

\begin{itemize}\setlength\itemsep{-0.3em}
	
	\item Processamento de resultados OCR \ref{contribution_ocr_posprocessing}
	
	\begin{itemize}\setlength\itemsep{-0.3em}
		\item Conversão de resultados OCR \ref{ocr_results_conversion}
		\item Debugging \ref{contribution_debugging}
		\item Análise de texto \ref{contribution_text_analyses}
		\item Limpeza de OCR Tree \ref{contribution_clean_ocr}
		\item Categorização de Blocos \ref{contribution_categorize_blocks}
		\item Divisão de blocos \ref{contribution_divide_blocks}
		\item Cálculo de ordem de leitura \ref{contribution_reading_order}
		\item Segmentação de resultados \ref{contribution_result_segmentation}
	\end{itemize}
	
	\item Processamento de imagem \ref{contribution_image_processing}
	\begin{itemize}\setlength\itemsep{-0.3em}
		\item Correção de ângulos de rotação
		\item Corte de sombra nas margens
		\item Binarização de imagem
		\item Identificação de delimitadores
		\item Identificação de imagens no documento
		\item Segmentação de documento
	\end{itemize}
	
	\item Processamento de texto \ref{contribution_text_processing}
	\begin{itemize}\setlength\itemsep{-0.3em}
		\item Limpeza de hifenização
		\item Geração de output
	\end{itemize}
		
		
\end{itemize}


\section{Processamento de resultados OCR}
\label{contribution_ocr_posprocessing}


O resultado da aplicação de OCR em imagens de baixa qualidade ou complexas como é o caso de jornais, é na generalidade propício a um output ruidoso e com vários defeitos em diferentes níveis, quer seja nas bounding boxes dos blocos, texto reconhecido com erros, ruído reconhecido como texto, blocos que deveriam ser separados, etc. 

%% TODO ilustracoes de resultados OCR ruidosos
%%% intersecoes e caixas dentro de caixas
%%% ruido considerado como texto

Deste modo, surgiu a oportunidade de adicionar ao \textit{toolkit} funcionalidades que fossem capazes de abordar estes problemas.

Esta secção trata particularmente de funcionalidades aplicadas sobre os resultados de OCR, i.e. após a aplicação de reconhecimento de caracteres, não tendo influencia no input desse procedimento.


\highlight{Conversão de resultados OCR}
\label{ocr_results_conversion}

Como base para a manipulação dos dados, como mencionado anteriormente, foi utilizada a estrutura de dados OCR Tree. Consequentemente, um conjunto de conversores desta classe foram desenvolvidos, nomeadamente:

\begin{itemize}
	\item JSON : de e para JSON. Inspirado no output de Tesseract para dicionário.
	\item HOCR : de e para HOCR. Standard para armazenamento de dados resultantes de OCR.
	\item Texto : para texto.
	\item MD : para markdown.
\end{itemize}


\highlight{Debugging}
\label{contribution_debugging}

Uma das características fundamentais de OCR é a sua capacidade para visualização fácil de resultados. Uma das mais úteis ferramentas de debugging geradas foi a reconstrução da OCR Tree sobre a imagem original.

\highlight{draw\_bounding\_boxes}[\normalsize]

\textbf{Descrição:} Desenha blocos de OCR Tree sob uma imagem.

\textbf{Argumentos:}
\begin{itemize}\setlength\itemsep{-0.3em}
	\item ocr\_results : OCR Tree com os blocos a desenhar
	\item img : imagem a ser modificada
	\item draw\_levels : lista dos níveis de nodos a desenhar
	\item conf : confiança mínima de texto dos blocos de nível de texto a desenhar
	\item id : flag para desenhar id do bloco, caso disponível
\end{itemize}


%% TODO ilustracao de resultado de drawing blocks

Outros métodos, mais circunstanciais, criados são para o desenho do template de um jornal e de artigos respetivamente.

\highlight{draw\_journal\_template}[\normalsize]

\textbf{Descrição:} Desenha o template de um jornal sob uma imagem.

\textbf{Argumentos:}
\begin{itemize}\setlength\itemsep{-0.3em}
	\item journal\_data : dicionário com entradas dos segmentos de um jornal (header, columns, footer). Cada um com um objeto do tipo Box que dita a bounding box do elemento. No caso das colunas é uma lista de Box.
	\item img : imagem a ser modificada
\end{itemize}

%% TODO ilustracao de resultado de drawing journal template

\highlight{draw\_articles}[\normalsize]

\textbf{Descrição:} Desenha artigos sob uma imagem.

\textbf{Argumentos:}
\begin{itemize}\setlength\itemsep{-0.3em}
	\item articles : lista de listas de OCR Tree. Assume-se que cada lista de OCR Tree é um artigo. Cada OCR Tree será um nodo de nível 2
	\item img : imagem a ser modificada
\end{itemize}

%% TODO ilustracao de resultado de drawing articles


\highlight{Análise de texto}
\label{contribution_text_analyses}

A análise de texto dos resultados OCR permite inferir características sobre o documento que não estão à partida disponíveis na OCR Tree. Foi desenvolvido um conjunto de métodos que cada um infere uma das seguintes métricas a partir da OCR Tree principal:

\begin{itemize}
	\item Tamanho de texto normal : método \textbf{get\_text\_sizes}
	\item Espaçamento médio de palavras : método \textbf{analyze\_text}
	\item Colunas do documento : método \textbf{get\_columns}
\end{itemize}

Todas estas métricas podem ser obtidas na chamada do método \textbf{analyze\_text}.

Naturalmente, a qualidade do cálculo destas métricas irá depender da qualidade dos resultados OCR, ou no caso de análise de texto, do nível de confiança de texto usado.

Os métodos \textbf{get\_text\_sizes} e \textbf{get\_columns} são ambos baseados em análise de frequências e procura de picos na curva destas.


\highlight{get\_text\_sizes}[\normalsize]

\textbf{Descrição:} Analisa as frequências dos tamanhos de linha, pesados pelo número de palavras que as respetivas linhas têm, de modo a obter os tamanhos de letra mais proeminentes por análise dos picos da curva de frequências.

A curva é obtida a partir de um smoothing da lista de frequências e os picos são calculados baseado em proeminência.

Devolve pelo menos um tamanho de letra : normal\_text\_size.

\textbf{Argumentos:}
\begin{itemize}\setlength\itemsep{-0.3em}
	\item ocr\_results : OCR Tree a analisar
	\item method : método de smoothing. Opções : WhittakerSmoother (por defeito), savgol\_filter
	\item conf : confiança de texto mínima. Restringe as palavras utilizadas para o cálculo do tamanho das linhas
\end{itemize}

\textbf{Algoritmo:}
\begin{breakablealgorithm}
	\caption{Cálculo de tamanhos de texto}
	\begin{algorithmic}[1]
		
		\State text\_sizes = \{
			'normal\_text\_size : 0
		\}
		
		\State lines $\leftarrow$ obtem linhas da OCR Tree
		\State line\_sizes = []
		\State \textit{// Cálculo das frequências de tamanhos das linhas}
		\For{line in lines}
			\If{line não é vazia e não é texto vertical}
				\State lmh $\leftarrow$ altura média da linha (arredondado a inteiro)
				\State linhe\_sizes[lmh] $\leftarrow$ soma 1 + peso (nº palavras de confiança na linha)
			\EndIf
		\EndFor
		
		\State \textit{// smoothing das linhas}
		\State \textit{// WhittakerSmoother : lambda = 1e1; order = 3; data\_length= len(line\_sizes)}
		\State \textit{// savgol\_filter : window\_length = round(len(line\_sizes*0.1)); polyorder = 2}
		\State line\_sizes\_smooth $\leftarrow$ smooth de line\_sizes usando o método escolhido
		
		\State \textit{// cálculo de picos utilizando a função find\_peaks do módulo spicy com proeminência a 10\% da frequência máxima}
		\State peaks\_smooth $\leftarrow$ cálculo dos picos
		
		\State text\_sizes['normal\_text\_size'] $\leftarrow$ máximo das frequências, i.e. altura mais comum
		
		\State \textit{// se houver mais picos, aqueles abaixo do maior serão tamanhos de letra pequena e da mesma forma para picos acima do maior}
		
	\end{algorithmic}
\end{breakablealgorithm}


\highlight{get\_columns}[\normalsize]

\textbf{Descrição:} Analisa as frequências das margens das bounding boxes dos blocos de nível 2, pesados pelo número de palavras que os respetivos blocos têm, de modo a obter os pontos esquerdos e direito mais proeminentes por análise dos picos das curvas de frequências.

A curva é obtida a partir de um smoothing da lista de frequências e os picos são calculados baseado em proeminência.

Devolve uma lista do tipo Box com o espaço das colunas encontradas.

Este método é bastante dependente da qualidade dos blocos reconhecidos, sendo que em geral é recomendado usar o método com o mesmo objetivo mas abordagem de análise de imagem, discutido na secção de processamento de imagem.

\textbf{Argumentos:}
\begin{itemize}\setlength\itemsep{-0.3em}
	\item ocr\_results : OCR Tree a analisar
	\item method : método de smoothing. Opções : WhittakerSmoother (por defeito), savgol\_filter
	\item conf : confiança de texto mínima. Restringe as palavras utilizadas para o peso das frequências das margens
\end{itemize}

\textbf{Algoritmo:} Algoritmo semelhante ao método get\_text\_sizes mas com análise das margens esquerda e direita das bounding boxes de nodos de nível 2. São calculados os picos de margens esquerdas e direitas separadamente que são depois pareados de forma a formar o espaço das diferentes colunas.



\highlight{Limpeza de OCR Tree}
\label{contribution_clean_ocr}

Como já descrito anteriormente, com a norma dos resultados da aplicação de OCR em documentos antigos, surge a oportunidade de criar um conjunto de métodos que ajudem numa "limpeza" geral dos resultados, de forma a, por exemplo: remover elementos de ruído; unir blocos com características semelhantes; separar blocos demasiado afastados; ajustar dimensões das bounding boxes, etc..


%% TODO ilustracao : exemplos dos problemas listados

Descrevem-se então alguns dos métodos propostos para resolver alguns destes problemas.


\highlight{block\_bound\_box\_fix}[\normalsize]

\textbf{Descrição:} Ajuste de bounding boxes para eliminar interseções.

\textbf{Argumentos:}
\begin{itemize}\setlength\itemsep{-0.3em}
	\item ocr\_results : OCR Tree a limpar
	\item text\_confidence : confiança de texto mínima. Utilizado para verificar se blocos são vazios
	\item find\_delimiters : flag que indica se verificação por delimitadores é realizada apenas por verificação do atributo "type" ou utilizando o método "is\_delimiter" de OCR\_Tree
	\item find\_images : flag que indica se verificação por imagens é realizada apenas por verificação do atributo "type" ou utilizando o método "is\_image" de OCR\_Tree
\end{itemize}

\textbf{Algoritmo:}

%% TODO : escrever algoritmo, depois de dar refactoring na funcao

%% TODO ilustracao : exemplo de uso do metodo para limpar blocos


\highlight{text\_bound\_box\_fix}[\normalsize]

\textbf{Descrição:} Ajuste de bounding boxes para apenas englobar o texto confiáveis dentro delas.

\textbf{Argumentos:}
\begin{itemize}\setlength\itemsep{-0.3em}
	\item ocr\_results : OCR Tree a ajustar
	\item text\_confidence : confiança de texto mínima. Utilizado para filtrar blocos de texto confiáveis
\end{itemize}

\textbf{Algoritmo:}

\begin{breakablealgorithm}
	\caption{Cálculo de tamanhos de texto}
	\begin{algorithmic}[1]
		
		\State blocks $\leftarrow$ obtém nodos de nível 2
		\State text\_blocks $\leftarrow$ filtra blocks para apenas ter blocos com texto
		\State \textit{// Percorre blocos com texto para ajustar as suas BBs}
		\For{b in text\_blocks}
			\State words $\leftarrow$ obtém lista de palavras confiáveis, não vazias
			\State block\_min\_left $\leftarrow$ valor da palavra mais à esquerda
			\State block\_max\_right $\leftarrow$ valor da palavra mais à direita
			\State block\_min\_top $\leftarrow$ valor da palavra mais acima
			\State block\_max\_bottom $\leftarrow$ valor da palavra mais abaixo
			
			\If{valores calculados ajustam caixa para ser menor}
				\State b.box $\leftarrow$ atualiza tamanho da caixa e dos seus filhos
			\EndIf
		\EndFor

		
	\end{algorithmic}
\end{breakablealgorithm}

%% TODO ilustracao : exemplo de uso do metodo para ajustar blocos



\highlight{remove\_solo\_words}[\normalsize]

\textbf{Descrição:} Remove blocos com uma única palavra que estão dentro de outros blocos. Estes são blocos que por esta característica podem ser identificados como ruído ou, em caso negativo, iriam causar conflito no output final por não estarem propriamente integrados no bloco devido.

\textbf{Argumentos:}
\begin{itemize}\setlength\itemsep{-0.3em}
	\item ocr\_results : OCR Tree a limpar
	\item conf : confiança de texto utilizada para análise a nível das palavras
\end{itemize}


\highlight{unite\_blocks}[\normalsize]

\textbf{Descrição:} Une verticalmente blocos com o mesmo valor de atributo "type". Apenas aplica união entre blocos adjacentes e horizontalmente alinhados. Nota : blocos de texto vertical, apenas podem unir com blocos de texto vertical.

\textbf{Argumentos:}
\begin{itemize}\setlength\itemsep{-0.3em}
	\item ocr\_results : OCR Tree a limpar
	\item conf : confiança de texto utilizada para verificação de blocos vazios
\end{itemize}

\textbf{Algoritmo:}
\begin{breakablealgorithm}
	\caption{União de blocos}
	\begin{algorithmic}[1]
		
		\State blocks $\leftarrow$ obtém nodos de nível 2
		\State target\_block $\leftarrow$ escolhe bloco não visitado
		\State \textit{// Percorre blocos com texto para ajustar as suas BBs}
		\While{não tiver visitado todos os blocos}
			\State united = False
			\State bellow\_blocks $\leftarrow$ lista de blocos adjacentes, diretamente abaixo
			\State bellow\_blocks $\leftarrow$ filtra pelos blocos do mesmo tipo que target\_block e horizontalmente alinhados
			\If{bellow\_blocks}
				\If{target\_block é texto vertical}
					\State bellow\_blocks $\leftarrow$ filtra para apenas manter texto vertical
				\EndIf
				
				\If{len(bellow\_blocks) == 1}
					\State target\_block $\leftarrow$ une com bloco
					\State ocr\_results $\leftarrow$ remove bloco unido
					\State \textit{// atualiza lista de não visitados}
				\EndIf
				
			\EndIf
			
			\State \textit{// Se não tiver unido, escolhe novo bloco, senão repete verificação}
			\If{not united}
				\State target\_block $\leftarrow$ escolhe próximo bloco não visitado
			\EndIf
		\EndWhile
		
		
	\end{algorithmic}
\end{breakablealgorithm}

%% TODO ilustracao :exemplo de uso do metodo para unir blocos



\highlight{split\_whitespaces}[\normalsize]

\textbf{Descrição:} Separa blocos com espaçamento entre palavras suficientemente grande. Apenas realiza divisão perpendicular ao eixo x. Tem em conta uma análise do texto para calcular o espaçamento médio de palavras e um valor de ratio de espaço vazio para realizar divisão, analisa a presença de espaços brancos em blocos com texto e divide-os quando as condições se verificam. No caso de blocos com múltiplas linhas, tem de verificar se existe um espaço vazio comum válido.

\textbf{Argumentos:}
\begin{itemize}\setlength\itemsep{-0.3em}
	\item ocr\_results : OCR Tree a limpar
	\item conf : confiança de texto utilizada para verificação de blocos vazios e análise de texto
	\item dif\_ratio : ratio de espaço entre palavras para realizar divisão
\end{itemize}

\textbf{Algoritmo:}

\begin{breakablealgorithm}
	\caption{Divisão por espaços vazios}
	\footnotesize
	\begin{algorithmic}[1]
		
		\State text\_analysis $\leftarrow$ analise do texto
		\State avg\_word\_dist = text\_analysis['average\_word\_distance']
		\State blocks $\leftarrow$ obtém blocos com texto
		
		\State \textit{// Percorre blocos com texto para ajustar as suas BBs}
		\For{block in blocks}
			\State lines $\leftarrow$ obtém linhas do bloco
			\State line\_seq\_positions = [] 
			\State valid\_split = True
			
			\State \textit{// Para cada linha procura espaços vazios válidos para divisão e guarda as coordenadas destes}
			\For{line in lines}
				\State line\_words $\leftarrow$ obtém palavras da linha
				\State line\_seq\_position = [None,None]
				\State line\_word\_dists $\leftarrow$ guarda as distancias entre palavras na linha
				\State line\_word\_pairs $\leftarrow$ guarda pares de palavras
				
				\If{line\_word\_dists}
					\State average $\leftarrow$ media entre a media do tamanho das palavras e o avg\_word\_dist
					\State line\_seq\_position $\leftarrow$ procura primeiro espaço que valide a regra $ dist \geq dif\_ratio * average$
					\If{not line\_seq\_positio}
						\State \textit{// todas as linhas têm de ter um espaço válido}
						\State valid\_split = False
					\Else
						\State line\_seq\_positions.append(line\_seq\_position)
					\EndIf
				\EndIf
				
			\EndFor
			
			\If{valid\_split and line\_seq\_positions}
				\State \textit{// verifica se todos os intervalos guardados intersetam}
				\State \textit{// em caso positivo, verifica o máximo tamanho da divisão a fazer}
				\State widest\_interval $\leftarrow$ espaço entre palavras mais longo
				\State interception $\leftarrow$ verifica se todas as linhas têm um espaço que está dentro de widest\_interval
				\If{interception}
					\State left $\leftarrow$ ponto mais à esquerda entre todos os espaços
					\State right $\leftarrow$ ponto mais à direita entre todos os espaços
					\State division\_line $\leftarrow$ Box vertical representante da linha de divisão
					\State \textit{// realizar divisão}
					\State blocks =  split\_block(block,delimiter,orientation='vertical',keep\_all=True,conf=conf)
					\State ocr\_results $\leftarrow$ atualiza OCR Tree, adicionando novo bloco resultante da divisão
				\EndIf
			\EndIf
		\EndFor
		
		
	\end{algorithmic}
\end{breakablealgorithm}



%% TODO ilustracao : exemplo de uso do metodo para separar usando whitespaces

\highlight{Categorização de blocos}
\label{contribution_categorize_blocks}

De forma a poder segmentar com maior precisão documentos, informação adicional sobre os blocos reconhecidos é necessária. Por exemplo, para ser possível identificar um artigo de um jornal, seria expectável agrupar um título e pelo menos uma caixa de texto. Desta forma uma tokenização dos blocos de nível 2 seguindo heurísticas foi desenvolvida.

O método que realiza esta tokenização é \textbf{categorize\_box}. O método foi desenvolvido com foco na utilização para jornais, sendo que os tipos de elementos identificados é portanto limitada.

A tokenização segue as seguintes regras:

\begin{itemize}\setlength\itemsep{-0.3em}
	\item \textbf{text} 		: bloco com texto, de tamanho dentro do tamanho normal de texto (margem de 10\%), ou que não verificou nenhuma das outras regras para blocos de texto
	\item \textbf{title} 	: bloco com texto, não vertical, com menos de 10 palavras e tamanho maior do que o tamanho normal de texto
	\item \textbf{highlight} : bloco com texto, semelhante a "title", mas com maior número de palavras
	\item \textbf{caption} 	: bloco com texto, de tamanho menor do que tamanho normal de texto e está diretamente abaixo de uma imagem ou de um bloco do tipo "caption"
	\item \textbf{delimiter} : bloco sem texto e que verifica a regra: $box.width >= 4*box.height \vee box.height >= 4*box.width$
	\item \textbf{other} 	: bloco sem texto e que não verificou nenhuma das outras regras para blocos vazios
\end{itemize}

Blocos de tipo "delimiter" ou "image" são também possíveis de identificar utilizando métodos disponibilizados para processamento de imagem.

No caso de categorização de blocos de texto, são também verificadas certas características:
\begin{itemize}\setlength\itemsep{-0.3em}
	\item \textbf{start\_text} : texto foi iniciado, i.e. começa com uma letra maiúscula ou com início de diálogo (sinais: ",',-) seguido de uma letra maiúscula.
	\item \textbf{ends\_text} : texto foi terminado, termina com um sinal de pontuação que finaliza uma frase, ou fim de diálogo (".","?","!",""","'")
\end{itemize}

Estas características de texto guardam informação relevante utilizada em métodos de cálculos de atração entre blocos.



%% TODO ilustracao : exemplo de uso do metodo para categorizar os blocos de um jornal


\highlight{Divisão de blocos}
\label{contribution_divide_blocks}

Uma funcionalidade de extrema utilidade para a manipulação dos produtos de OCR é a capacidade de dividir um bloco em dois, cada um destes novos ficando com parte dos conteúdos do bloco original. Tal operação facilita a remoção de partes ruidosas de um bloco, ou a segmentação de bloco quando este foi incorretamente unido pelo software de OCR. Exemplo: bloco de texto cuja primeira linha tem potencial de ser título, permite facilmente dividir o bloco em o título e restante texto.

%% TODO ilustracao : exemplo de divisao de bloco, para separar titulo do texto

Esta é no entanto uma operação de relativa complexidade, visto que é necessário ir descendo na árvore e fazer múltiplas verificações para calcular o melhor destino final (entre os novos blocos) para cada um dos filhos.

A divisão implementada admite dois tipos de corte: horizontal, utilizando um corte perpendicular ao eixo y, tem uma divisão do conteúdo relativamente simples, visto apenas ser necessário verificar quais linhas se mantêm em cada novo bloco, refazendo posteriormente os parágrafos conforme necessário; vertical, utilizando um corte perpendicular ao eixo x, tem a divisão de conteúdo mais complexo, visto ser necessário verificar para cada palavra a que bloco pertence e, posteriormente, refazer as linhas e os parágrafos. 

Uma aplicação direta desta funcionalidade será discutida sobre o editor OCR, ferramenta que permite uma visualização mais fácil desta transformação.


\highlight{split\_block}[\normalsize]

\textbf{Descrição:} Dado um bloco, um delimitador representante do corte a realizar e a orientação do corte pretendida, divide o bloco em 2. Reparte o conteúdo original entre os dois novos blocos de acordo com a área em que melhor se incluem. Dos blocos criados, devolve apenas aqueles que ainda tiverem conteúdo (texto).

\textbf{Argumentos:}
\begin{itemize}\setlength\itemsep{-0.3em}
	\item block : OCR Tree a dividir
	\item delimiter : Box representante do corte a ser feito
	\item orientation : orientação do corte a realizar. Opções: "horizontal", "vertical".
	\item keep\_all : flag que indica se todo o conteúdo deve ser restaurado. Em caso negativo, conteúdo que não se inclui totalmente em nenhum dos novos blocos não será incluído. Se positivo, conteúdo em posição de conflito irá para o bloco que possui a maior parte da sua área.
\end{itemize}

\textbf{Algoritmo:}

\begin{breakablealgorithm}
	\caption{Divisão de bloco em 2}
	\footnotesize
	\begin{algorithmic}[1]
		
		\State new\_blocks = [block]
		
		\If{delimiter não interceta com bloco}
			\Return new\_blocks
		\EndIf
		
		\State area\_1, area\_2 $\leftarrow$ cria 2 Box para os novos blocos, criadas de acordo com a orientação do corte e a posição do delimitador
		
		\State \textit{// listas para parágrafos de cada área}
		\State blocks\_1 = []
		\State blocks\_2 = []
		
		\State \textit{// corte horizontal}
		\State \textit{// obtém todas as linhas e verifica a que área pertencem}
		\State \textit{// reconstrói parágrafos no fim}
		\If{orientation == "horizontal"}
			\State lines $\leftarrow$ obtém nodos de nível 4
			\State area\_1\_lines, area\_2\_lines = []
			\For{line in lines}
				\If{line dentro de area\_1}
					\State area\_1\_lines.append(line)
				\ElsIf{line dentro de area\_2}
					\State area\_2\_lines.append(line)
				\ElsIf{keep\_all}
					\If{line está mais incluída dentro da área 1}
						\State area\_1\_lines.append(line)
					\Else
						\State area\_2\_lines.append(line)
					\EndIf
				\EndIf
			\EndFor
			
			\If{area\_1\_lines}
				\State blocks\_1 $\leftarrow$ reconstrói parágrafos de acordo com a metadata (par\_num) dos nodos linha
			\EndIf
			
			\If{area\_2\_lines}
			\State blocks\_2 $\leftarrow$ reconstrói parágrafos de acordo com a metadata (par\_num) dos nodos linha
			\EndIf
		
		\Else
			\State \textit{// corte vertical}
			\State \textit{// obtém todas as linhas e para cada uma verifica que palavras pertencem a qual área}
			\State block $\leftarrow$ da id às palavras para as conseguir remover diretamente
			\State blocks\_1 $\leftarrow$ copia paragrafos pertence ao bloco
			\State blocks\_2 $\leftarrow$ copia paragrafos pertence ao bloco
			\For{paragrafo in block}
				\State par\_words $\leftarrow$ obtém palavras do parágrafo
				\For{word in par\_words}
					\If{word dentro de area\_1}
						\State blocks\_2 $\leftarrow$ remove palavra da area 2
					\ElsIf{word dentro de area\_2}
						\State blocks\_1 $\leftarrow$ remove palavra da area 1
					\ElsIf{keep\_all}
						\If{word está mais incluída dentro da área 1}
							\State blocks\_2 $\leftarrow$ remove palavra da area 2
						\Else
							\State blocks\_1 $\leftarrow$ remove palavra da area 1
						\EndIf
					\EndIf
				\EndFor
				\State \textit{// atualiza BB dos parágrafos para representar as modificações}
			\EndFor
		\EndIf
		
		\If{blocks\_1}
			\State block.children = blocks\_1
			\State block $\leftarrow$ atualiza BB para representar modificações do conteúdo
			
		\Else
			\State \textit{// se a área 1 não tiver blocos, bloco original é atualizado com a área 2 e só é devolvido um bloco}
			\If{blocks\_1 vazio}
				\State block.children = blocks\_2
				\State block $\leftarrow$ atualiza BB para representar modificações do conteúdo
			\Else
				\State \textit{// cria novo bloco}
				\State new\_block $\leftarrow$ nova OCR Tree
				\State new\_block.children = blocks\_2
				\State new\_block $\leftarrow$ atualiza BB para representar modificações do conteúdo
				\State new\_blocks = [block,new\_block]
				
			\EndIf
			
		\EndIf
		
		\Return new\_blocks
		
	\end{algorithmic}
\end{breakablealgorithm}



%% TODO ilustracao : exemplo de divisao de bloco, vertical e horizontal

\highlight{Cálculo de ordem de leitura}
\label{contribution_reading_order}

No caso de documentos de leitura não linear, como é o caso de jornais, a ordem de leitura proposta pelos motores OCR pode muitas vezes ser incorreto.

%% TODO ilustracao : ordem inicial errada de um jornal

Foi então desenvolvido um algoritmo geral, baseado em grafos pesados, para calcular a ordem de leitura do corpo de um jornal. Este algoritmo procura, através de um conjunto de regras sobre as características dos blocos dos resultados OCR, como por exemplo: texto inacabado, texto seguido de título; calcular a atração entre blocos de acordo com as suas características e posição relativa (similar a ordenação topológica); e posteriormente ordenar os blocos, procurando agrupar artigos. Um artigo considera-se como um grupo de blocos, em que o primeiro é um título e não pode ter mais títulos dentro do grupo, a não ser que seja diretamente a seguir ao primeiro título (complementares).

Para conseguir este produto final foi então necessário criar uma estrutura de dados do tipo grafo, neste caso pesado; um método para calcular a atração entre nodos do grafo; e o algoritmo que calcula o caminho do grafo.

\highlight{Estrutura de dados graph}[\normalsize]

A estrutura de dados representante do grafo é composta pelas classes:

\begin{itemize}\setlength\itemsep{-0.3em}
	\item Graph : representante do grafo geral
	
		Possuí as funcionalidades principais:
		\begin{itemize}\setlength\itemsep{-0.3em}
			\item verificações de existência de um nodo
			\item verificação de existência de caminho entre nodos
			\item limpeza de caminhos transitivos no grafo
			\item limpeza de caminhos de baixo peso : em nodos com múltiplos pais, se a atração para um pai é menos de metade do que a atração para outro, remove aresta
		\end{itemize}
	\item Node : representante de um nodo do grafo
		Possuí as funcionalidades principais:
		\begin{itemize}\setlength\itemsep{-0.3em}
			\item conjunto de pais
			\item conjunto de filhos
			\item atualizar pais e filhos
			\item cálculo de conexão de um nodo a outro
			\item limpeza de caminhos de baixo peso : em nodos com múltiplos pais, se a atração para um pai é menos de metade do que a atração para outro, remove aresta
		\end{itemize}
	\item Edge : representante de um aresta do grafo
\end{itemize}


\highlight{Cálculo de atração entre blocos}[\normalsize]

O cálculo da atração entre dois blocos tem em conta as características dos dois blocos, assim como o contexto da sua vizinhança. O resultado final é o somatório de pontuações adquiridas pela verificação de um conjunto de regras, com diferentes níveis de importância. 

Sendo o conjunto de verificações relativamente extenso, segue-se um compilado das condições verificadas, por ordem de relevância:

\textbf{Atração}
\begin{itemize}\setlength\itemsep{-0.3em}
	\item bloco 'image' tem muita atração a bloco 'caption' (vice-versa)
	\item bloco 'title' é muito atraído a bloco não 'title'
	\item bloco 'text' inacabado é muito atraído por bloco 'text' não começado diretamente por baixo dele
	\item bloco é bastante atraído para blocos à sua direita
	\item bloco 'text' inacabado é bastante atraído por bloco 'text' não começado diretamente à direita dele
	\item bloco 'title' é bastante atraído por bloco 'text' com texto começado
	\item bloco é atraído por blocos abaixo dele
	\item bloco é atraído por bloco à direita se bloco diretamente abaixo abrange os dois blocos
	\item bloco 'title' é atraído por bloco não 'title'
	\item bloco 'title' é atraído por bloco 'title' apenas se estiver diretamente abaixo dele
	\item bloco 'text' é atraído por bloco abaixo mais esquerda
	\item bloco 'text' é atraído por bloco abaixo
	\item bloco é um pouco atraído pelo bloco por baixo mais à esquerda
	\item bloco é um pouco atraído pelo bloco à direita se o bloco mais à esquerda abaixo não tiver texto
\end{itemize}

\textbf{Repulsão}
\begin{itemize}\setlength\itemsep{-0.3em}
	\item bloco não é um pouco atraído pelo bloco abaixo se o bloco mais à esquerda abaixo não tiver texto
	\item bloco 'text' não acabado não é atraído por bloco não 'text'
	\item bloco 'text' não acabado não é atraído por bloco 'text' começado
\end{itemize}

O método responsável por este cálculo é \textbf{calculate\_block\_attraction}.

\textbf{Argumentos:}
\begin{itemize}\setlength\itemsep{-0.3em}
	\item block : OCR Tree cuja atração por target\_block será calculada
	\item target\_block : OCR Tree a comparar
	\item blocks : lista de OCR Tree para servirem de contexto de vizinhança
	\item direction : direção entre block e target\_block, se nenhuma for dada, direção é calculada. Utilizado para cálculo de distância e verificação de condições. Opções : 'above', 'left','right','below'
	\item child : flag que indica se cálculo é de pai para filho ou vice-versa
\end{itemize}



\highlight{Cálculo da ordem de leitura}[\normalsize]

O cálculo da ordem de leitura é realizado pelo método \textbf{sort\_topologic\_order} que utiliza um algoritmo de ordenação topológica que utiliza os pesos das arestas para resolver conflitos.

Para escolher o primeiro bloco, ou quando não há mais blocos abaixo do bloco atual, utiliza o método next\_top\_block que procura dentro de uma lista de potências próximos blocos, o mais apropriado - normalmente o com menor distância ao canto superior esquerdo. 

\textbf{Argumentos:}
\begin{itemize}\setlength\itemsep{-0.3em}
	\item topologic\_graph : Graph com as ligações dos blocos da OCR Tree a ordenar
	\item sort\_weight : flag que indica se ordenação é apenas topológica ou usa os pesos do grafo para resolver situações de desempate
\end{itemize}



%% TODO ilustracao : calculo da ordem de leitura de alguns jornais

Os resultados desta ordenação é bastante dependente da qualidade do conteúdo da OCR Tree, sendo que uma OCR Tree com texto muito disperso ou ruidoso irá provavelmente gerar uma ordenação errada. 

Diferentes jornais podem não ter métodos de leitura tão simples, ou que acomodem estas regras, sendo que esta metodologia não será ubíqua. 



\highlight{Segmentação de resultados}
\label{contribution_result_segmentation}

No caso de jornais, como já discutido anteriormente, o elemento base da informação nestes contido são os artigos. 

Artigos podem ser descritos como um título seguido por um excerto de elementos de outro tipo, por exemplo texto ou imagem, quebrado pelo final de página ou por um novo título que iniciaria outro artigo. Seguindo esta caracterização foi desenvolvida um método que, utilizando os resultados do cálculo da ordem de leitura i.e., o grafo e a ordem de nodos calculada, procura formar uma lista de artigos.

\highlight{graph\_isolate\_articles}[\normalsize]

\textbf{Descrição:} A partir de um grafo de nodos de blocos e uma ordem de leitura destes, procura percorrer esta lista para gerar artigos utilizando todos os blocos. Um artigo é começado por um bloco título e, se um novo bloco título for iterado na lista e a lista tiver blocos não título, inicia um novo artigo.

\textbf{Argumentos:}
\begin{itemize}\setlength\itemsep{-0.3em}
	\item graph : Graph com as ligações dos blocos da OCR Tree
	\item order\_list : list de ordem de leitura dos blocos da OCR Tree
\end{itemize}


%% TODO ilustracao : segmentacao de jornal em artigos

Métodos para conversão da lista de artigos para texto, markdown e visualização sob uma imagem foram também desenvolvidos.



\section{Processamento de imagem}
\label{contribution_image_processing}


Um outro ponto de abordagem natural quando se trabalha com OCR, é o tratamento do input, que na maior parte dos casos se trata de uma imagem. O estado deste input inicial terá a maior influência na qualidade dos resultados ao longo de todo o processo.

Os principais problemas que se procurou abordar neste trabalho em relação ao tratamento de imagens de documento foram: correção automática de rotações do documento em relação ao texto; remoção de sombras de bainha nas bordas do documento; métodos de binarização para documentos antigos e em mau estado; segmentação de documentos, focando em jornais (divisão entre header e body, divisão de colunas); identificação de imagens/ilustrações no documento; identificação de delimitadores/linhas divisórias no documento. 

Além disso, trabalho já realizado para a resolução de problemas como: upscaling, denoising e ajustes na iluminação de imagem; foi reaproveitado, embora maioritariamente para o uso na pipeline desenvolvida (\ref{contribution_toolkit_pipeline}).

%% TODO ilustracao : imagens de documentos antigos em mau estado, aplicação de alguns métodos para melhorar estado

Esta secção irá abordar os métodos desenvolvidos neste sentido.


\highlight{Correção de ângulo de rotação}
\label{contribution_image_rotation_correction}

Para imagens de alta definição, i.e. normalmente com dpi elevado e, portanto, mais favorável para realização de OCR, notou-se que mesmo pequenas inclinações no texto podem resultar em resultados de OCR extremamente ruidosos e desorganizados.

%% TODO : ilustracao de pequenas rotacoes e resultado depois de OCR

Foi então desenvolvido um algoritmo com o propósito de automaticamente detetar a existência de uma rotação do texto e, calculando o seu ângulo, corrigi-la. 

Este algoritmo foi baseado maioritariamente na solução proposta por \cite{4283429} (discutida na secção de estado da arte) com adaptações adicionadas para complementar pontos não explicados da solução e para ser mais viável para documentos com algum ruído.

Na sua essência o algoritmo procura, num documento com texto, analisar as posições do primeiro pixel preto (expectável ser a primeira instância de texto à esquerda) à esquerda em diferentes linhas, para tentar desenhar uma linha, na qual a inclinação será calculada. As adaptações realizadas serviram para fazer uma decisão inicial sobre a direção da rotação, e para, assumindo que se trabalha com imagens com ruído, procurar a linha mais confiável para o cálculo da inclinação.

Para ajustes mais minuciosos, também é aplicado o método de ajuste de rotação da biblioteca leptonica. Este método por si só não é capaz de corrigir ângulos mais consideráveis (+- acima de 6 graus).


%% TODO : ilustracao no estilo do paper para analise dos pixeis

\highlight{calculate\_rotation\_direction}[\normalsize]

\textbf{Descrição:} Calcula a direção de rotação num documento de texto. Retorna um dos valores: 'counter-clockwise'; 'clockwise'; 'none'. Cria conjuntos de pontos alinhados (que possibilitem traçar uma reta) para cada uma das direções e, de acordo com qual das direções tem o maior conjunto retorna o identificador dessa direção.

\textbf{Argumentos:}
\begin{itemize}\setlength\itemsep{-0.3em}
	\item image : imagem do documento a analisar
	\item line\_quantetization : valor de quantetização do eixo y da imagem a usar (para a análise dos pixeis) 
\end{itemize}

\highlight{rotate\_image}[\normalsize]

\textbf{Descrição:} Corrige rotação numa imagem de um documento, se houver. Retorna a imagem com a rotação aplicada. Dada uma direção de rotação, calcula os conjuntos de pixeis de texto que formam linhas e com essa dada inclinação no documento. Escolhe o maior destes conjuntos para formar uma linha e calcular a sua inclinação. Aplica uma rotação na imagem para corrigir a inclinação detetada. 
Se possível, aplica o método \textit{pixFindSkewAndDeskew} de leptonica para correções menores.

\textbf{Argumentos:}
\begin{itemize}\setlength\itemsep{-0.3em}
	\item image : imagem do documento a analisar
	\item line\_quantetization : valor de quantetização do eixo y da imagem a usar (para a análise dos pixeis) 
	\item direction : direção da rotação. 'counter-clockwise', 'clockwise' ou 'auto'. Se auto, aplica o método acima para calcular a direção.
	\item auto\_crop : flag para aplicação do método de cropping de possível bainha na imagem. Estas quando presentes afetam consideravelmente o algoritmo pois não costumam estar alinhadas com o texto.
\end{itemize}

\highlight{Corte de sombra nas margens}
\label{contribution_image_cut_argin_shadow}

No caso de documentos com múltiplas folhas, como livros ou jornais, é comum a presença de uma sombra nas margens das páginas. Esta pode, como mencionado anteriormente, impactar a solução para correção de rotação de documento, mas também, quando aplicado OCR, esta sombra poderá ser detetada como um elemento (um tipo de delimitador ou figura) que na realidade gera ruído extra. Pode ser então útil a remoção desta secção da imagem.

%% TODO : ilustracao de documento com sombra da bainha

O algoritmo de remoção segue uma lógica semelhante aos métodos de análise de tamanhos de texto ou de deteção de colunas anteriormente explicados. Na imagem binarizada, é analisada a frequência de pixeis pretos na imagem e, se nas suas extremidades esquerda ou direta (10\% de cada uma das pontas) for detetado um pico seguido de uma descida muito acentuada (igual ou menor a 10\% da frequência máxima de pontos pretos) é recortado a zona desse pico da imagem.

\highlight{cut\_document\_margins}[\normalsize]

\textbf{Descrição:} Procura cortar sombras de bainha de uma imagem de um documento. Se estas não existirem, recortam margem da imagem para se aproximar do texto, se este for adjacente a uma zona vazia. Retorna uma instância do tipo Box com as dimensões da imagem recortada.

\textbf{Argumentos:}
\begin{itemize}\setlength\itemsep{-0.3em}
	\item image : imagem do documento a analisar
	\item method : método de smoothing. Opções : WhittakerSmoother (por defeito), savgol\_filter
\end{itemize}


\highlight{Binarização de imagem}
\label{contribution_image_binarization}

Como foi discutido no estado da arte sobre o processamento de imagens, não existe um método de binarização que funcione de forma geral para qualquer tipo de documento, visto que muitas vezes estes são adaptativos e diferentes distribuições dentro da imagem vão gerar resultados diferentes.

Assim sendo, o trabalho realizado neste sentido não foi com o objetivo de propor um método generalista, mas sim de decidir métodos úteis no contexto do trabalho.

Com o trabalho realizado, conclui-se que para documentos com pouco ruído, ou onde este se apresente distribuído pela imagem sem grandes focos particulares, um denoising de médias não locais seguido de uma binarização com algoritmo de Otsu, obtém resultados satisfatórios. Esta binarização é similar à realizada pelo Tesseract.
No caso de o ruído estar focado num local específico da imagem, verificou-se que binarizações mais extremas no tresholding, como estilo fax, conseguem obter melhor resultado nas zonas mais afetadas.

%% TODO : ilustracao 2 binarizacoes em casos de documento sem ruido focado e com ruido focado. mostrar resultado de ocr

Um método para cada uma destas binarizações foi desenvolvido.

\highlight{binarize}[\normalsize]

\textbf{Descrição:} Binarização utilizando tresholding de Otsu. Realiza limpeza de ruído utilizando algoritmo de médias não locais.

\textbf{Argumentos:}
\begin{itemize}\setlength\itemsep{-0.3em}
	\item image : imagem a binarizar
	\item denoise\_strength : valor de intensidade a utilizar para denoising
	\item invert : flag para inverter cor de binarização
\end{itemize}

\highlight{binarize\_fax}[\normalsize]

\textbf{Descrição:} Binarização utilizando algoritmo de emulação de fax. Algoritmo de emulação proposto pelo orientador Prof. José João. Utilizando métodos do image magick pode ser conseguido como : convert <image> -colorspace Gray ( +clone -blur 15,15 ) -compose Divide\_Src -composite -level 10\%,90\%,0.2.


\textbf{Argumentos:}
\begin{itemize}\setlength\itemsep{-0.3em}
	\item image : imagem a binarizar
	\item g\_kernel\_size : tamanho do kernel gaussiano utilizado para o cálculo da imagem desfocada
	\item g\_sigma : valor de sigma do kernel gaussiano utilizado para o cálculo da imagem desfocada
	\item black\_point : valor de tresh de ponto preto, utilizado para ajustar os níveis da imagem
	\item white\_point : valor de tresh de ponto branco, utilizado para ajustar os níveis da imagem 
	\item gamma : valor usado para ajuste do gamma após ajustar níveis da imagem
	\item is\_percentage : flag que indica se valores de black e white point são para serem considerados como percentagem
	\item invert : flag para inverter cor de binarização
\end{itemize}

\textbf{Algoritmo:}

Calcula-se primeiramente uma versão desfocada da imagem. Esta será usada para se criar uma distinção entre o plano de fundo e o texto da imagem, ao se dividir a imagem original pela desfocada. Utiliza-se 2 kernels gaussianos por defeito, um aplicado a cada eixo, de dimensão 15 e sigma 15.

Em seguida ajusta-se os níveis dos valores pretos e brancos da imagem de forma a estarem contidos dentro dos valores definidos, 10\% e 90\% da escala disponível respetivamente. 

Por fim, ajusta-se o gamma da imagem para a escurecer.


\highlight{Identificação de delimitadores}
\label{contribution_image_delimiter_identification}

Para documentos estruturados como jornais, certos elementos são utilizados de forma geral para realçar a dada estrutura, como é o caso de delimitadores.

%% TODO : ilustracao de delimitadores de um jornal

Pode então ser útil conseguir identificar estes elementos para facilitar o seu uso, por exemplo, no cálculo de ordem de leitura de um jornal e cálculo de atração entre blocos.

Estes delimitadores são, por norma, linhas verticais ou horizontais, podendo ser filtradas utilizando um algoritmo similar ao desenvolvido pelo OpenCV \citep{cv_extract_lines}. Este algoritmo é utilizado para separar linhas horizontais do resto da imagem. Adaptando para o propósito de identificar as linhas horizontais, é necessário, após separar as linhas do resto da imagem, utilizar o algoritmo de Hough para identificar as posições e dimensões das linhas existentes.

\highlight{get\_document\_delimiters}[\normalsize]

\textbf{Descrição:} Identifica e devolve uma lista de delimitadores em uma imagem. Delimitadores são linhas horizontais ou verticais. Lista devolvida são objetos do tipo Box.


\textbf{Argumentos:}
\begin{itemize}\setlength\itemsep{-0.3em}
	\item image : imagem a analisar
	\item tmp\_dir : diretório usado para armazenar ficheiros temporários
	\item min\_length\_h : valor de comprimento mínimo para um delimitador horizontal válido
	\item min\_length\_v : valor de comprimento mínimo para um delimitador vertical válido
	\item max\_line\_gap\_h : valor de descontinuidade mais comprida aceite para um delimitador horizontal válido
	\item max\_line\_gap\_v : valor de descontinuidade mais comprida aceite para um delimitador vertical válido
	\item reduce\_delimiters : flag para reduzir/limpar lista de delimitadores, procurando remover ruído
\end{itemize}

\textbf{Algoritmo:}


\begin{breakablealgorithm}
	\caption{Identificação de delimitadores (apenas horizontais)}
	\footnotesize
	\begin{algorithmic}[1]
		
		\State image $\leftarrow$ imagem binarizada
		
		\State morph $\leftarrow$ aplica dilatação geral, seguido de linhas horizontais, aplica-se filtro de estruturas para filtrar linhas horizontais e, por fim, aplica-se nova dilatação para acentuar as linhas
		
		\State edges $\leftarrow$ método Canny para deteção de contornos, seguido de acentuação destes por dilatação
		
		\State horizontal\_lines = cv2.HoughLinesP(edges,1,np.pi/100,50,None,minLineLength=min\_length\_h,maxLineGap=max\_line\_gap\_h)
		
		\State lines = horizontal\_lines
		
		\State delimiters = []
		
		\For{line in lines}
			
			\State \textit{// verifica se linhas são verticais ou horizontais}
			\State \textit{// compara razão entre variação de x ou de y (declives da reta) com tan(60) e tan(5) respetivamente}
			\State is\_vertical = dy != 0 and abs(dx/dy) < t5
			\State is\_horizontal = dx != 0 and abs(dy/dx) < t60
			
			\If{is\_vertical or is\_horizontal}
				\State delimiters $\leftarrow$ adiciona Box para linha
			\EndIf
		\EndFor
		
		\State delimiters $\leftarrow$ remove delimitadores na borda (provavelmente sombras)
		
		\If{reduce\_delimiters}
			\State delimiters $\leftarrow$ aplica limpezas sobre lista: remove delimitadores com muitas interseções, une delimitadores próximos e com a mesma direção, remove delimitadores com muitas componentes ligadas (provavelmente texto)
		\EndIf
		
		\State delimiters $\leftarrow$ remove delimitadores demasiado curtos (menos de 5\% do tamanho da imagem)
		
	\end{algorithmic}
\end{breakablealgorithm}


Na generalidade dos casos, a quantidade de linhas detetadas será muito superior aos delimitadores esperados, devido ao o ruído, texto do documento e morfologias aplicadas. 
Diferentes valores de parâmetros para o algoritmo de Hough variam estes resultados.

%% TODO : ilustracao de linhas identificadas, antes de reduzir

É então recomendado uma aplicação de métodos para reduzir o número de delimitadores encontrados.
Os que foram desenvolvidos foram:

\begin{itemize}\setlength\itemsep{-0.3em}
	\item \textbf{remover interseções} : se 2 delimitadores com diferente orientação intersetam, remove o menor
	\item \textbf{une delimitadores} : une delimitadores de igual orientação e suficientemente próximos
	\item \textbf{remover linhas com demasiadas componentes ligadas} : linhas com muitas componentes ligadas, são provavelmente texto, ruído ou outro tipo de elemento, sendo portanto removidas
\end{itemize}

%% TODO : ilustracao de linhas identificadas, depois de reduzir



\highlight{Identificação de imagens no documento}
\label{contribution_image_document_image_identification}

% imagens

A possibilidade de identificar e isolar elementos de imagem/ilustração em documentos é, por si só, uma ferramenta extremamente útil. Infelizmente nem todos os motores OCR identificam diretamente estes elementos como sendo imagens, por exemplo, no caso do Tesseract imagens são rodeadas com boa precisão porém, no ficheiro resultante de OCR, não é atribuído um identificador para isolar o elemento como uma imagem (de igual modo para os delimitadores). 

Assim sendo, explorou-se abordagens para se obter esta informação de forma mais focada.
Entre estas explorou-se abordagens apoiadas por Deep Learning, como é o caso de Detectron desenvolvido pela Meta, e o \href{https://layout-parser.readthedocs.io/en/latest/index.html}{Layout Parser} que surge a partir do Detectron. 
Uma solução mais simples em termos de dependências/instalação e computação é o uso de um método de segmentação de página disponibilizado pela biblioteca Leptonica (\href{https://tpgit.github.io/Leptonica/livre__pageseg_8c_source.html}{método de segmentação}). 
Este método identifica, através de múltiplas aplicações de morfologias, bounding boxes para zonas previstas de texto e imagem.
O código original foi apenas levemente modificado para alterar o destino do output.


%% TODO : ilustracao de imagens identificadas

\highlight{identify\_document\_images}[\normalsize]

\textbf{Descrição:} Método que invoca algoritmo de leptonica para segmentação de página, interpreta o output deste, e retorna uma lista de objetos do tipo Box que representa a posição das imagens identificadas no documento.


\textbf{Argumentos:}
\begin{itemize}\setlength\itemsep{-0.3em}
	\item image : imagem a binarizar
	\item tmp\_dir : diretório para armazenar ficheiros temporários
\end{itemize}


\highlight{Segmentação de documento}
\label{contribution_image_segmentation}

Como discutido na secção de métodos aplicados sob os resultados de OCR, a segmentação de um documento tem diversas aplicações. No entanto, como também foi mencionado, a segmentação utilizando os resultados de OCR é muito dependente da qualidade destes, sendo preferível procurar realizar esta diretamente através da imagem.

Para este propósito, foram desenvolvidos métodos para identificação de colunas e para segmentação de documento (header,body,footer).

\highlight{divide\_columns}[\normalsize]

\textbf{Descrição:} Método que, através de análise da distribuição de pixeis pretos numa imagem de um documento, divide este em colunas. Retorna uma lista de objetos do tipo Box.
É recomendado remover ilustrações/imagens do documento pois estes são aglomerados que são outliers na análise de frequências.

\textbf{Argumentos:}
\begin{itemize}\setlength\itemsep{-0.3em}
	\item image : imagem a binarizar
	\item method : método de smoothing. Opções : WhittakerSmoother (por defeito).
\end{itemize}

\textbf{Algoritmo:} 

Similar ao utilizado para a deteção de colunas através dos blocos de resultado de OCR, porém a análise de frequências é sobre a distribuição de pixeis invés das margens dos blocos.


%% TODO : ilustracao de divisao de documento em colunas


\highlight{segment\_document}[\normalsize]

\textbf{Descrição:} Método que, obtendo uma lista de delimitadores detetados no documento, divide procura dividir o documento em header,footer e body.


\textbf{Argumentos:}
\begin{itemize}\setlength\itemsep{-0.3em}
	\item image : imagem a binarizar
	\item remove\_images : flag para aplicar método de remoção de imagens
	\item tmp\_dir : diretoria de ficheiros temporários
	\item target\_segments : lista segmentos de interesse. Opções possíveis: 'header', 'body','footer'
\end{itemize}

\textbf{Algoritmo:} 

Utilizando uma lista de delimitadores horizontais, procura ajustar o body, inicialmente considerado como a imagem inteira, em header e footer (se procurados). 

O header é assinalado por um delimitador horizontal que segue a seguinte regra:

$delimiter.bottom \leq 0.3*image.shape[0] \wedge delimiter.width \geq 0.4*image.shape[1]$

O footer é assinalado por um delimitador horizontal que segue a seguinte regra:

$delimiter.bottom \geq 0.7*image.shape[0] \wedge delimiter.width \geq 0.4*image.shape[1]$


%% TODO : ilustracao de divisao de documento em segmentos



\section{Processamento de texto}
\label{contribution_text_processing}

% resoluçao de hifenização


\highlight{Geração de output}

A criação de outputs finais de fácil visualização é uma função essencial do trabalho produzido, permitindo uma digestão dos resultados de OCR mais simples.

Fundamentalmente, existem dois métodos principais de geração de output:

\begin{itemize}\setlength\itemsep{-0.3em}
	\item A partir de uma OCR Tree, chamar o método da classe "to\_text" para transformar a árvore numa única string de texto. A ordem do texto seguirá a ordem dos filhos de cada nodo. Delimitadores personalizados para cada um dos níveis podem ser distribuídos através do fornecimento de um dicionário.
	\item A partir da classe Article, invocar o método "to\_text" para transformar a árvore interna processada numa string simples do título seguido do corpo do artigo. 
	
	Invocar o método "to\_md" para gerar uma string markdown que utiliza uma sintaxe mais apropriada para markdown, adicionando também separador entre o título e o corpo. Neste caso, se o artigo tiver elementos de imagem, é possível que estes sejam incluídos no output final.
\end{itemize}







\chapter{OSDOCR Pipeline - Implementação}
\label{cap_osdocr_pipeline_implementacao}

Neste capítulo, será explorada a implementação da componente de pipeline.

De forma a criar um caso de uso para as ferramentas criadas e outras exploradas e disponibilizadas, foi criado um comando de aplicação de uma pipeline de aplicação de OCR, que permite, a partir de uma imagem: aplicar tratamentos para melhorar a sua qualidade; aplicar OCR; tratar os resultados; e gerar um output simples, ou específico para jornais.

Através dos argumentos passados à ferramenta, a pipeline consegue comportar-se de forma diferente, por exemplo: aceitar como output inicial uma instância de Hocr ou de OCR Tree em formato JSON; ignorar pré-processamento ou passos específicos deste; etc.

As funcionalidades disponíveis na pipeline são uma culminação do trabalho descrito anteriormente sobre o OSDOCR Toolkit, assim como ferramentas exploradas para resolver questões que este não aborda como, por exemplo, realização de upscaling de uma imagem.

\section{Sumário}

O comando de invocação da pipeline é 'osdocr'. Este apresenta as seguintes opções de utilização:

\begin{itemize}\setlength\itemsep{-0.8em}
	\item \textbf{target} 
		\begin{itemize}\setlength\itemsep{-0.5em}
			\item \textbf{alternativo} : t
			\item \textbf{descrição} : ficheiro de imagem a ser processado pela pipeline.
		\end{itemize}
		
	\item \textbf{file}
		\begin{itemize}\setlength\itemsep{-0.5em}
			\item \textbf{alternativo} : f
			\item \textbf{descrição} : ficheiro de texto que lista um target por linha.
		\end{itemize}
		
	\item \textbf{target\_ocr\_results} 
		\begin{itemize}\setlength\itemsep{-0.5em}
			\item \textbf{alternativo} : tocr
			\item \textbf{descrição} : ficheiro de resultados OCR. Pode ser de tipo hocr ou json. Se fornecido, não será realizado pré processamento de imagem ou OCR.
		\end{itemize}
		
	\item \textbf{output\_folder}
		\begin{itemize}\setlength\itemsep{-0.5em}
			\item \textbf{alternativo} : of
			\item \textbf{descrição} : caminho para guardar os resultados
		\end{itemize}
		
	
	\item \textbf{segmented\_ocr}
		\begin{itemize}\setlength\itemsep{-0.5em}
			\item \textbf{alternativo} : sgocr
			\item \textbf{descrição} : flag para aplicação de OCR no target será realizada em cada um dos segmentos, invés de na imagem inteira. Cada uma das partes é posteriormente unida para criar uma única OCR Tree de resultados.
			\item \textbf{valor default} : False
		\end{itemize}
		
	\item \textbf{target\_segments}
		\begin{itemize}\setlength\itemsep{-0.5em}
			\item \textbf{alternativo} : ts
			\item \textbf{descrição} : segmentos a calcular no target. Segmento body é sempre obtido. O body é ainda repartido nas suas colunas.
			\item \textbf{opções} : 'header', 'body', 'footer'
			\item \textbf{valor default} : 'header','body'
		\end{itemize}
		
	\item \textbf{force\_ocr}
		\begin{itemize}\setlength\itemsep{-0.5em}
			\item \textbf{alternativo} : focr
			\item \textbf{descrição} : flag que ignora possíveis ficheiros em cache de resultados OCR, consequentes de iterações anteriores do target. Força aplicação de na imagem.
			\item \textbf{valor default} : False
		\end{itemize}
		
	\item \textbf{tesseract\_config}
		\begin{itemize}\setlength\itemsep{-0.5em}
			\item \textbf{descrição} : flags para serem passadas ao Tesseract no momento de aplicação de OCR. Estas estão disponíveis na documentação do Tesseract. Cada argumento tem de ter como prefixo '\_\_' para permitir o seu processamento.
			\item \textbf{valor default} : \_\_l por
		\end{itemize}
	
	\item \textbf{text\_confidence}
		\begin{itemize}\setlength\itemsep{-0.5em}
			\item \textbf{alternativo} : tc
			\item \textbf{descrição} : valor de confiança de texto que pipeline irá usar.
			\item \textbf{valor default} : 10
		\end{itemize}
	
	
	\item \textbf{split\_whitespace}
		\begin{itemize}\setlength\itemsep{-0.5em}
			\item \textbf{alternativo} : sw
			\item \textbf{descrição} : valor utilizado como razão entre um espaço branco num bloco de texto e a média pesada dos espaçamentos entre palavras, para este ser considerado válido como ponto de divisão de um bloco.
			\item \textbf{valor default} : 3
		\end{itemize}
	
	\item \textbf{fix\_rotation}
		\begin{itemize}\setlength\itemsep{-0.5em}
			\item \textbf{alternativo} : fr
			\item \textbf{descrição} : opções usadas para a correção de rotação de documento.
			\item \textbf{opções} : 'auto','clockwise','counter-clockwise'
			\item \textbf{valor default} : 'auto'
		\end{itemize}
	
	\item \textbf{upscaling\_image}
		\begin{itemize}\setlength\itemsep{-0.5em}
			\item \textbf{alternativo} : upi
			\item \textbf{descrição} : opções usadas para o upscaling de imagem. Se opção 'autoscale' do modelo 'waifu2x' for escolhido, aplica upscaling da imagem até esta chegar ao dpi alvo (opção target\_dpi).
			\item \textbf{opções} : 'waifu2x' -> 'scale2x', 'scale4x', 'autoscale'
			\item \textbf{valor default} : 'waifu2x'
		\end{itemize}
	
	\item \textbf{target\_dpi}
		\begin{itemize}\setlength\itemsep{-0.5em}
			\item \textbf{alternativo} : tdpi
			\item \textbf{descrição} : valor do dpi alvo
			\item \textbf{valor default} : 300
		\end{itemize}
	
	\item \textbf{target\_dimensions}
		\begin{itemize}\setlength\itemsep{-0.5em}
			\item \textbf{alternativo} : tdim
			\item \textbf{descrição} : valor das dimensões físicas a usar para cálculo do dpi da imagem. Opções estão disponíveis no ficheiro 'dimensions.json' no projeto, podendo ser atualizado.
			\item \textbf{opções} : 'A5', 'A4', 'A3', 'A2', 'A1', 'A0', '2A0'
			\item \textbf{valor default} : A3
		\end{itemize}
	
	\item \textbf{denoise\_image}
		\begin{itemize}\setlength\itemsep{-0.5em}
			\item \textbf{alternativo} : di
			\item \textbf{descrição} : opções usadas para o denoising de imagem.
			\item \textbf{opções} : 'waifu2x' -> '-1' - '3'
			\item \textbf{valor default} : 'waifu2x'
		\end{itemize}
	
	\item \textbf{light\_correction}
		\begin{itemize}\setlength\itemsep{-0.5em}
			\item \textbf{alternativo} : lc
			\item \textbf{descrição} : opções usadas para a correção de iluminação de uma imagem.
			\item \textbf{opções} : 'best\_SSIM', 'best\_PSNR', 'LOL-Blur', 'SICE', 'SID', 'w\_perc'
			\item \textbf{valor default} : 'best\_SSIM'
		\end{itemize}
		
	\item \textbf{light\_correction\_split\_image}
	\begin{itemize}\setlength\itemsep{-0.5em}
		\item \textbf{alternativo} : lcs
		\item \textbf{descrição} : flag para aplicação de correção de iluminação da imagem em patches, invés de na totalidade, de modo a melhorar tempo de processamento. Para certos modelos, resulta em contrastes consideráveis entre os patches.
		\item \textbf{valor default} : True
	\end{itemize}
	
	\item \textbf{binarize\_image}
		\begin{itemize}\setlength\itemsep{-0.5em}
			\item \textbf{alternativo} : bi
			\item \textbf{descrição} : opções usadas para a binarização de imagem para aplicação de OCR.
			\item \textbf{opções} : 'fax', 'otsu'
			\item \textbf{valor default} : 'fax'
		\end{itemize}
	
	\item \textbf{remove\_document\_images}
		\begin{itemize}\setlength\itemsep{-0.5em}
			\item \textbf{alternativo} : bi
			\item \textbf{descrição} : método utilizado para remoção de imagens.
			\item \textbf{opções} : 'leptonica', 'layoutparser'
			\item \textbf{valor default} : 'leptonica'
		\end{itemize}
	
	\item \textbf{target\_old\_document}
		\begin{itemize}\setlength\itemsep{-0.5em}
			\item \textbf{alternativo} : tod
			\item \textbf{descrição} : flag para indicar que target é um documento antigo. Utilizado quando método 'layoutparser' é escolhido, de forma a escolher o modelo mais apropriado
			\item \textbf{valor default} : True
		\end{itemize}
	
	\item \textbf{ignore\_delimiters}
		\begin{itemize}\setlength\itemsep{-0.5em}
			\item \textbf{alternativo} : igd
			\item \textbf{descrição} : flag para ignorar delimitadores. Se ativada, estes não serão tidos em conta como indicadores do layout do documento no cálculo da ordem de leitura.
			\item \textbf{valor default} : False
		\end{itemize}
	
	\item \textbf{skip\_method}
		\begin{itemize}\setlength\itemsep{-0.5em}
			\item \textbf{descrição} : métodos/passos da pipeline a ignorar.
			\item \textbf{opções} : 'leptonica', 'layoutparser'
			\item \textbf{valor default} : 'all', 'auto\_rotate', 'noise\_removal', 'blur\_removal', 'light\_correction', 'image\_preprocess', 'remove\_document\_margins', 'remove\_document\_images', 'image\_upscaling', 'identify\_document\_delimiters', 'binarize\_image', 'clean\_ocr', 'split\_whitespace', 'unite\_blocks', 'calculate\_reading\_order', 'extract\_articles', 'posprocessing'
		\end{itemize}
	
	\item \textbf{calibrate}
		\begin{itemize}\setlength\itemsep{-0.5em}
			\item \textbf{descrição} : aplicar modo de calibração de pipeline. Usado para encontrar melhor configuração de pipeline para um dado target. Pode ser dado um diretório onde estarão disponíveis os ficheiros necessários para calibração, e onde ficarão os resultados de calibração; e um diretório com configurações de pipeline a testar. Por defeito, o diretório procurado dá-se por 'calibration' no local onde o comando foi corrido, e são usadas configurações de pipeline disponíveis no projeto. 
		\end{itemize}
	
	\item \textbf{calibrate\_no\_reuse}
		\begin{itemize}\setlength\itemsep{-0.5em}
			\item \textbf{descrição} : flag para não utilizar cache existente.
			\item \textbf{valor default} : False
		\end{itemize}
	
	\item \textbf{pipeline\_config}
		\begin{itemize}\setlength\itemsep{-0.5em}
			\item \textbf{descrição} : ficheiro do tipo JSON com configuração de pipeline a usar. Pode ser usado como alternativa a aplicar argumentos no terminal de comandos.
		\end{itemize}
		
	\item \textbf{gui}
		\begin{itemize}\setlength\itemsep{-0.5em}
			\item \textbf{descrição} : aplicar modo de interface gráfica. Interface gráfica simples que pode ser utilizada para experimentar algumas das funcionalidades disponíveis. Maioritariamente usada para debugging.
		\end{itemize}
\end{itemize}


\section{Pré-processamento de imagem}

Como primeiro procedimento da pipeline, no caso de uso de inputs de imagem, tem-se o tratamento desta, para procurar obter, a partir da aplicação OCR, uma melhor transcrição do conteúdo textual, assim como melhor identificação de outros elementos, como figuras.

Para isso, procurou-se abordar os seguintes problemas: imagens rodadas; remoção de margens/sombras na margem de documentos; imagens de baixa resolução; remoção de figuras de documentos; imagens com ruído; imagens com distribuição de iluminação inconsistente;.

As solução destes problemas envolveram o uso de métodos desenvolvidos no toolkit, assim como soluções já existentes.

A ordem de aplicação destas soluções mostrou ser relevante pois estas podem interferir umas com as outras, por exemplo: aplicação de denoising antes de remover as figuras do documento pode afetar a identificação destas. Assim sendo, a ordem ótima de execução segue a listagem que se segue.

\highlight{Correção de rotação}

A correção de possíveis rotações de imagens de documentos é resolvida utilizando os métodos de correção de rotação descritos na secção de métodos de processamento de imagem.

\highlight{Remoção de sombras das margens do documento}

De igual modo, a remoção de possíveis sombras na margem de um documento aproveita os métodos desenvolvidos e descritos para este efeito na mesma secção. 

A aplicação desta funcionalidade deve vir após a correção de possíveis rotações pois, como esta solução envolve a análise da distribuição dos valores de pixeis, se estas sombras se apresentarem inclinadas a possibilidade que elas não sejam resolvidas aumenta.


\highlight{Upscaling de imagem}

Para a resolução de problemas resultantes de imagens com baixos valores de dpi, modelos de Deep Learning obtém resultados mais interessantes em relação a algoritmos passados, como interpolação. Dentro destes modelos, o \href{https://github.com/nagadomi/waifu2x}{waifu2x} é bastante respeitado no que toca a aumento de resolução de imagem e limpeza de imperfeições. 

Embora não especificamente focado para o processamento de imagens de documentos, o seu fácil uso e instalação, assim como a verificação de bons resultados mesmo nesta área relativamente externa ao treino, levaram a que este modelo seja uma boa adição às ferramentas de pré-processamento de imagem disponíveis na pipeline.

Esta funcionalidade, quando escolhida para realizar upscaling automático, é dependente das configurações utilizadas para cálculo de dpi de imagem (dimensões reais proporcionadas).

A aplicação de OCR é recomendada para dpi no valor de 300 ou superior, sendo que para imagens antigas, esta funcionalidade é essencial.

%% TODO : ilustracao de upscaling. zoom numa parte de imagem com e sem upscaling


\highlight{Remoção de ilustrações de documento}

De forma a diminuir a presença de elementos não identificados nos resultados OCR, assim como permitir identifica-los corretamente, esta funcionalidade procura remover as ilustrações de um documento, recortando-as para uma pasta temporário, permitindo a sua reposição na imagem final.

Entre as soluções exploradas para esta questão, a que se enquadrava mais na questão de identificação de ilustrações de documentos, e até com atenção a jornais, foi o modelo \href{https://layout-parser.readthedocs.io/en/latest/index.html}{Layout Parser}, mais especificamente, aproveitando o modelo Detectron desenvolvido pela Meta.
Este, no entanto, apresenta uma instalação complexa devido a dependências de bibliotecas da própria Meta que demonstram incompatibilidades e que necessitam ser modificadas manualmente.

Deste modo, aproveitou-se uma alternativa com resultados também satisfatórios e de menor custo computacional, com métodos de segmentação de documento propostos pela Leptonica. Estes foram descritos também na secção anterior.

Comparando os dois, a opção de usar Leptonica é satisfatória na generalidade dos casos, até identificando com maior precisão as ilustrações (delimitando menos fundo do documento) em vários casos de estudo. Os métodos de Leptonica tendem no entanto a errar para ilustrações que não apresentem uma borda notável e tenham um fundo similar ao fundo do documento. Nestes casos, o Layout Parser é mais propício a acertar.

%% TODO : ilustracao de ilustrações removidas de um documento

É também importante realçar que a pipeline, no processo de recortar as ilustrações, preenche o espaço vazio com uma média das cores de fundo da imagem. Este passo é importante pois caso contrário a binarização posterior da imagem, dependendo da distribuição das cores desta, pode resultar numa imagem inutilizável.

%% TODO : ilustracao de binarizacao nos dois casos de remocao de ilustracoes

\highlight{Denoising de imagem}

Para a realização de denoising, foi também aplicado o modelo de \textbf{waifu2x}, visto este proporcionar opções para este efeito. Este denoising é realizado em imagens a cor, sendo portanto subtil.

O denoising mais relevante é realizado na binarização realizada antes da aplicação de OCR.

%% TODO : ilustracao de denoising.


\highlight{Correção de iluminação}

Para a correção de iluminação de documentos, também se conclui que modelos de Deep Learning são a melhor opção. A família de modelos que mostraram resultados mais interessantes foi \href{https://github.com/Fediory/HVI-CIDNet}{HVI-CIDNet}, Os pesos para estes estão disponibilizados nesse mesmo repositório.

Para imagens de maior resolução, o tempo de processamento destes modelos é considerável, sendo que como solução a pipeline apresenta uma opção para dividir a imagem em patches e correr o modelo em cada um destes, unindo-os no final. Para alguns modelos, esta divisão pode resultar em contrastes notáveis na imagem

%% TODO : ilustracao de correcao de iluminacao utilizando diferentes modelos

%% TODO : ilustracao de correcao de iluminacao com e sem patches


\section{OCR}

% OCR de imagem
%% binarizacao
%%% imagem simples
%%% imagem segmentada

\section{Pós-processamento de OCR}

% tratamento de resultados
%% clean OCR
%% categorizacao OCR
%% Uniao OCR
%% Find titles TODO : nao falado anteriormente

\section{Geração de output}

% criacao de output
%% output para jornal
%%% calculo da ordem de leitura
%%% output para text
%%% output para markdown
%% output simples
%%% output text

\section{Validação de resultados}

% validacao de resultados/pipeline






\chapter{OSDOCR Editor - Implementação}
\label{cap_osdocr_editor_implementacao}


% proposito e ferramentas utilizadas

% funcionalidades disponiveis
%% abir diferentes resultados OCR para uma mesma imagem
%% mover blocos singularmente ou em grupo
%% ajustar dimensoes de blocos
%% OCR de bloco
%%% configuracoes de OCR
%% configuracoes de editor
%% dividir bloco manualmente
%% dividir bloco por espaços vazios
%% categorizar blocos manualmente
%% ordenar blocos
%% calcular ordem de blocos
%% gerar artigos automaticamente
%% gerir artigos
%%% adicionar e remover
%%% adicionar e remover blocos a artigos
%%% alterar ordem dos artigos
%% gerar output
%%% simples
%%% markdown
%%% OCR Tree
%% cache de operacoes
%%% retroceder e reconstruir operacoes





% \section{GUI Simples}
% \label{gui_simples}
% 
% De forma a facilitar a visualização dos resultados dos motores OCR, assim como das transformações realizadas nestes pelas diferentes técnicas aplicadas, foi implementado um GUI simples em Python utilizando a biblioteca PySimpleGUI. Isto tornou o processo de análise dos dados mais intuitiva e interativa, principalmente no processo de manipulação de blocos.
% 
% O formato da interface gráfica é relativamente simples, servindo principalmente o uso de debugging. Esta permite:
% 
% \begin{itemize}\setlength\itemsep{0.05cm}
%     \item Escolha de ficheiro de input - ficheiros imagem
%     \item Aplicação de reconhecimento na imagem - utilizando Tesseract
%     \item Visualizar blocos dos resultados
%     \item Visualizar texto de bloco
%     \item Aplicar funcionalidades e visualizar resultados
%     \begin{itemize}
%         \item Limpeza de blocos
%         \item Ordenação de caixas
%         \item Extração de artigos
%         \item Cálculo de template de jornal utilizando delimitadores
%     \end{itemize}
% \end{itemize}
% 
% Seguem-se alguns exemplos da interface:
% 
% \subsubsection{Interface gráfica}
% 
% \begin{figure}[H]
%     \centering
%     \includegraphics[width=1\textwidth]{images/implementacao/gui/gui_base.png}
%     \caption{Interface gráfica simples}
%     \label{fig:gui_base}
% \end{figure}
% 
% \subsubsection{Visualização de bounding boxes}
% 
% \begin{figure}[H]
%     \centering
%     \includegraphics[width=1\textwidth]{images/implementacao/gui/gui_draw_bb.png}
%     \caption{Visualização dos blocos resultantes de OCR}
%     \label{fig:gui_draw_bb}
% \end{figure}
% 
% A visualização de blocos dispõe também de coloração diferente para os blocos de acordo com a sua categorização. Blocos título estão a azul escuro, texto a azul claro, delimitadores a vermelho, legendas a branco e o resto - imagens, outros - a verde.
% 
% \subsubsection{Cálculo de template de jornal}
% 
% \begin{figure}[H]
%     \centering
%     \includegraphics[width=1\textwidth]{images/implementacao/gui/gui_draw_template.png}
%     \caption{Visualização do cálculo do template de jornal}
%     \label{fig:gui_draw_template}
% \end{figure}
% 
% O cálculo de template é feito através da deteção e análise dos delimitadores dos resultados OCR. Áreas são depois calculadas de acordo com estes delimitadores e, como se pode ver no caso do Header (caixa a verde) da imagem \ref{fig:gui_draw_template}, a área é ajustada de acordo com as caixas com texto da respetiva área.
% 
% \subsubsection{Extração de artigos}
% 
% \begin{figure}[H]
%     \centering
%     \includegraphics[width=1\textwidth]{images/implementacao/gui/gui_draw_articles.png}
%     \caption{Visualização dos artigos extraídos}
%     \label{fig:gui_draw_article}
% \end{figure}
% 
% Neste caso, os artigos são calculados e posteriormente escolhidas cores distintas para realçar cada um destes. Os artigos são representados pelo conjunto de blocos que foram agrupados como sendo um dado artigo.
% 
% \subsubsection{Limpeza de bounding boxes}
% 
% \begin{figure}[H]
%     \centering
%     \includegraphics[width=1\textwidth]{images/implementacao/gui/gui_fix_blocks.png}
%     \caption{Visualização da limpeza de blocos}
%     \label{fig:gui_fix_bb}
% \end{figure}
% 
% Para facilitar a deteção das diferenças entre o antes e depois da limpeza, os dois estados são postos lado a lado e os blocos são identificados, mantendo a mesma identificação após a limpeza. 





\chapter{Resultados}
\label{cap_results}

Nesta secção será realizado um estudo sobre os resultados e competências da solução implementada.

Como discutido previamente, a solução desta tese é composta por 3 módulos distintos, cada um deles tendo sido explorado em capítulos dedicados, mas dos quais 2 - OSDOCR Pipeline e OSDOCR Editor - são dependentes e produtos do 1º, sendo este, OSDOCR Toolkit, a base do projeto como um todo.

Desta forma, será dedicada uma secção para cada uma destas componentes, com especial tratamento para a primeira pois, sendo um caso particular de implementação que dificulta uma apreciação sucinta, pode ser tomada como seus resultados, os próprios resultados das outras duas componentes.

As duas restantes componentes terão também por sua vez métricas de apreciação distintas, que serão nas respetivas secções explicadas.


\section{OSDOCR Toolkit}


O toolkit para melhoria do processo de OCR serviu como objetivo base do atual trabalho. Este, através do estudo do estado da arte, reflexão sobre a filosofia da solução, assim como pelo cíclico processo de implementação e teste, evoluiu para englobar diferentes aspetos da aplicação de OCR: desde o processamento de imagem; tratamento, análise e manipulação dos resultados de OCR; e tratamento de texto.
Dentro destes foi evidente o foco na segunda componente do processo, sendo a que menor atenção tem nas soluções correntes, que tendem a focar no processamento de imagem ou tratamento de texto fazendo uso de deep learning.

Existem diferentes formas de avaliar um toolkit, como: a apreciação da capacidade das suas ferramentas de forma individual; a sua utilidade em contextos além dos designados para a sua utilização; e a capacidade de integração em soluções mais complexas, i.e. soluções que façam uso do toolkit.

Ao longo do capítulo \ref{cap_osdocr_toolkit_implementacao}, à medida que eram expostas as diferentes ferramentas desenvolvidas, exemplos do uso destas foram sido fornecidos, estando uma demonstração individual destas já fornecido.

A solução engloba ainda 2 outras componentes que, como base, fazem uso deste mesmo toolkit. Estas 2 componentes, ambas também com o foco concreto de uso em jornais antigos, podem ser utilizadas em contextos extra, como revistas ou documentos atuais. Assim, através da avaliação destas 2 soluções, podemos simultaneamente, embora indiretamente, avaliar o Toolkit.

% dificil de verificar resultados devido a metodos nao produzirem diretamente resultados

% verificar aplicacao de alguns metodos especificos

\section{OSDOCR Pipeline}


\subsection{Metodologia}


A pipeline de aplicação de OCR é, das 3 componentes da solução, a que mais facilmente pode ser objetivamente discutida. Esta é das 3 a que se assemelha mais aos trabalhos envolvendo aplicação de OCR, onde se tem um resultado direto e comparável com um valor inicial, i.e. texto real de um documento (ground truth), do qual se podem obter diferentes métricas, comparado com o texto resultante da aplicação de OCR numa imagem.

Seguindo um processo semelhante, obter-se-ão então resultados da aplicação desta componente sob diferentes casos de teste. 

Relativamente aos casos de teste escolhidos, estes constam um total de 18 diferentes jornais, para um total de 35 páginas a serem testadas. O número amostras é inferior ao ideal sendo limitado pelo escopo do projeto - com uma solução envolvendo 3 componentes, todas de considerável dimensão -; à necessidade de, na generalidade dos casos, ser necessária uma cuidada e manual transcrição das amostras escolhidas; e múltiplas utilizações de cada amostra para diferentes resultados.

As amostras escolhidas apresentam, de modo a testar diferentes capacidades da pipeline, características particulares que as permitem caracterizar em :

\begin{itemize}\setlength\itemsep{-0.9em}
	\item \textbf{Template compacto (4)} : templates com muito texto, compactado, e normalmente resolução abaixo da desejada; 
	\item \textbf{Ordem de leitura complexa (2)} : templates que apresentam uma ordem de leitura não linear (cima para baixo, esquerda para a direita), fazendo uso do contexto do texto ou de guias (ex.: delimitadores) para guiar o leitor;
	\item \textbf{Moderno (1)} : jornais modernos, com resolução e estado ideal;
	\item \textbf{Inclinados (2)} : jornais que foram digitalizados com má orientação;
	\item \textbf{Template simples (4)} : Jornais que apresentam um template e, consequentemente, uma ordem de leitura simples. Texto/colunas bem espaçados;
	\item \textbf{Outros (5)} : outros tipos de documentos como: revistas, jornais infantis com muitas ilustrações, banda desenhada, etc.;
\end{itemize}

Para cada um destes documentos, uma transcrição foi preparada manualmente, assim como ficheiros adicionais com transcrição parcial, e com características do documento como o número de artigos, imagens e colunas. 
Tal deve-se á escolha do módulo de validação da pipeline para obtenção e comparação de resultados. Este, como descrito na secção correspondente, procura comparar através de diferentes métricas uma verdade de um documento (ground truth) com os resultados obtidos de diferentes configurações da pipeline.

As métricas que nos focaremos para a tabulação de resultados serão:

\begin{itemize}\setlength\itemsep{-0.9em}
	\item Confiança média de texto
	\item Similaridade do texto (similaridade de cosseno)
	\item Rácio de aparições de palavras na GT e no resultado
	\item Rácio de palavras únicas na GT e no resultado
	\item Rácio de acerto da GT parcial no resultado
	\item Ordem geral da GT parcial no resultado
	\item Rácio de número de blocos de texto no resultado
\end{itemize}



\begin{figure}[H]
	\centering
	\includegraphics[width=0.7\textwidth]{images/ilustracoes/casos_teste.png}
	\caption{Amostras do conjunto de casos de teste.}
	\label{fig:casos_teste}
\end{figure}


Sendo a pipeline mutável através da configuração a ela passada, diferentes configurações foram desenhadas para possibilitar a extrapolação de mais conclusões. Note-se que esta lista não é extensiva, sendo que estas são multiplicativas ao número de execuções total.

Como base, será usada uma configuração de pipeline que não fará uso de nenhum módulo desta para além de OCR.

As outras configurações são, de forma sucinta:

\begin{enumerate}\setlength\itemsep{-0.9em}
	\item \textbf{Pipeline completa}: utilizada para testar uma sequência de todos os módulos implementados.
	\item \textbf{Pipeline completa sem ordenação}: possibilitando a comparação da ordenação original com a aplicação dos módulos.
	\item \textbf{Apenas pré-processamento}: permitindo a avaliação dos módulos de imagem na transcrição
	\item \textbf{Apenas pós-processamento}: afetando principalmente a capacidade de melhorar a leitura da transcrição geral
	\item \textbf{Pipeline geral}: configuração que, ao longo do trabalho, se avaliou como sendo mais ubíquo (relativamente)
	\item \textbf{Segmentação de imagem (e configuração geral)}: útil principalmente para casos de muito ruído ou texto compactado, procurando auxiliar a segmentação dos blocos.
\end{enumerate}

Como características gémeas entre estas configurações, temos que todas utilizam o tesseract como motor de OCR, utilizando a configuração de linguagem 'por' para português visto ser esta a linguagem dos casos de teste. No caso de ser realizado upscaling da imagem, procura-se aumentar os dpi para 150, assumindo uma dimensão de folha A3, ocorrendo para a generalidade dos casos o upscaling (exceção a categoria moderna). Por último, todas utilizam confiança de texto mínima 10.

No total, serão analisadas 245 (7 configurações x 35 páginas) execuções da pipeline.



Todos os preparativos para esta análise (jornais, ground truths e configurações de pipeline), assim como os resultados pré tabulados, estão disponíveis \href{https://drive.google.com/drive/u/0/folders/1DW-AIuSxjEyv6ioq7jX8P1xruy03Sxo9}{aqui}.


\subsection{Resultados}

Segue-se uma redução dos resultados das métricas escolhidas para análise da componente. Estes serão apresentados na forma de gráficos de barras, sendo que, como mencionado, a totalidade dos dados está também disponibilizada com métricas extra não aqui focadas.

Versões ampliadas dos gráficos estão disponíveis na secção de apêndices.

\begin{figure}[H]
	\centering
	\hspace*{-2cm}
	\includegraphics[width=1.2\textwidth]{images/resultados/graph_avg_text_conf.png}
	\caption{Valores de confiança média de texto das diferentes pipelines.}
	\label{fig:graph_avg_text_conf}
\end{figure}


\begin{table}[H]
	\centering
	\begin{tabular}{|l|c|}
		\hline
		\textbf{Pipeline} & \textbf{Média} \\ \hline
		simples & 85.296 				   \\ \hline
		completa & 84.847 				   \\ \hline
		completa (sem ordenação) & 84.847  \\ \hline
		apenas pré-proc. & 83.513 		   \\ \hline
		apenas pós-proc. & 86.335 		   \\ \hline
		geral & 85.859 					   \\ \hline
		segmentação & 84.759 			   \\ \hline
	\end{tabular}
	\caption{Média geral de confiança de texto das pipelines.}
\end{table}


Analisando os dados desta métrica, verifica-se que, na generalidade, não houve grandes oscilações na confiança média de texto, sendo que a pipeline geral e a de pós processamento obtiveram o melhor resultado médio, seguidos pela simples. Estas duas pipelines fazem dotam-se de fazer uso controlado, ou nenhum de processamento de imagem, aproveitando os processos de limpeza do toolkit para remover texto sem confiança, acoplado com melhorias de imagem.

Por outro lado, tendo a pior média e maior oscilação a pipeline de pré processamento, realça-se a questão que já se tinha anteriormente estudado neste trabalho, que a utilização indiscriminada de ferramentas de processamento de imagem não é benéfica, sendo preferível estas serem adaptadas para cada documento. 


\begin{figure}[H]
	\centering
	\hspace*{-2cm}
	\includegraphics[width=1.2\textwidth]{images/resultados/graph_gt_similiraty_cosine.png}
	\caption{Valores de similaridade do texto (similaridade por cosseno) com GT das diferentes pipelines.}
	\label{fig:graph_gt_similiraty_cosine}
\end{figure}


\begin{table}[H]
	\centering
	\begin{tabular}{|l|c|}
		\hline
		\textbf{Pipeline} & \textbf{Média} \\ \hline
		simples & 0.959 				   \\ \hline
		completa & 0.939 				   \\ \hline
		completa (sem ordenação) & 0.940   \\ \hline
		apenas pré-proc. & 0.945 		   \\ \hline
		apenas pós-proc. & 0.949 		   \\ \hline
		geral & 0.941 					   \\ \hline
		segmentação & 0.924 			   \\ \hline
	\end{tabular}
	\caption{Média geral de similaridade do texto com GT.}
\end{table}



\begin{figure}[H]
	\centering
	\hspace*{-2cm}
	\includegraphics[width=1.2\textwidth]{images/resultados/graph_gt_word_hit_ratio.png}
	\caption{Rácios de aparição total de palavras da GT das diferentes pipelines.}
	\label{fig:graph_gt_word_hit_ratio}
\end{figure}


\begin{table}[H]
	\centering
	\begin{tabular}{|l|c|}
		\hline
		\textbf{Pipeline} & \textbf{Média} \\ \hline
		simples & 0.855 				   \\ \hline
		completa & 0.787 				   \\ \hline
		completa (sem ordenação) & 0.797   \\ \hline
		apenas pré-proc. & 0.829 		   \\ \hline
		apenas pós-proc. & 0.806 		   \\ \hline
		geral & 0.808 					   \\ \hline
		segmentação & 0.787 			   \\ \hline
	\end{tabular}
	\caption{Média geral de rácios de aparição total de palavras da GT.}
\end{table}




\begin{figure}[H]
	\centering
	\hspace*{-2cm}
	\includegraphics[width=1.2\textwidth]{images/resultados/graph_gt_unique_word_hit_ratio.png}
	\caption{Rácios de aparição de palavras distintas da GT das diferentes pipelines.}
	\label{fig:graph_gt_unique_word_hit_ratio}
\end{figure}


\begin{table}[H]
	\centering
	\begin{tabular}{|l|c|}
		\hline
		\textbf{Pipeline} & \textbf{Média} \\ \hline
		simples & 0.860 				   \\ \hline
		completa & 0.801 				   \\ \hline
		completa (sem ordenação) & 0.809   \\ \hline
		apenas pré-proc. & 0.839 		   \\ \hline
		apenas pós-proc. & 0.814 		   \\ \hline
		geral & 0.820 					   \\ \hline
		segmentação & 0.797 			   \\ \hline
	\end{tabular}
	\caption{Média geral de rácios de aparição de palavras distintas da GT.}
\end{table}


\begin{figure}[H]
	\centering
	\hspace*{-2cm}
	\includegraphics[width=1.2\textwidth]{images/resultados/graph_text_block_ratio.png}
	\caption{Rácios de número de blocos de texto relativo à pipeline simples.}
	\label{fig:graph_text_block_ratio}
\end{figure}


\begin{table}[H]
	\centering
	\begin{tabular}{|l|c|}
		\hline
		\textbf{Pipeline} & \textbf{Média} \\ \hline
		completa & 0.635 				   \\ \hline
		completa (sem ordenação) & 0.635   \\ \hline
		apenas pré-proc. & 1.175 		   \\ \hline
		apenas pós-proc. & 0.487 		   \\ \hline
		geral & 0.658 					   \\ \hline
		segmentação & 0.747 			   \\ \hline
	\end{tabular}
	\caption{Média de rácio de número de blocos de texto comparado com pipeline simples.}
\end{table}




Verificando as métricas respetivas á ground truth, nota-se que em geral os resultados são mais favoráveis para o uso da template simples no que toca a métricas de similaridade simples de texto. Localmente, no entanto, podemos verificar que em certas instâncias, por exemplo, no documento 'o\_portuguez\_3', o rácio de palavras únicas detetadas melhorou cerca de 11\% com a pipeline geral relativamente ao OCR simples, e 12\% de acerto geral das palavras, com a generalidade das pipelines sendo melhor do que a base.

Por outro lado, assumindo que, embora na generalidade inferior, as diferentes configurações se assemelhem à base, nota-se a diferença na capacidade de reduzir a quantidade de dados da OCR Tree, i.e. quantidade de blocos nos resultados finais. Isto apenas se nota, naturalmente, nas pipelines que apresentam pós processamento. A diminuição da quantidade de blocos finais constitui uma redução da complexidade do resultado, permitindo uma análise mais fácil deste.



\begin{figure}[H]
	\centering
	\hspace*{-2cm}
	\includegraphics[width=1.2\textwidth]{images/resultados/graph_pgt_hit_ratio.png}
	\caption{Rácios de aparição de linhas da Partial GT das diferentes pipelines.}
	\label{fig:graph_pgt_hit_ratio}
\end{figure}


\begin{table}[H]
	\centering
	\begin{tabular}{|l|c|}
		\hline
		\textbf{Pipeline} & \textbf{Média} \\ \hline
		simples & 0.618 				   \\ \hline
		completa & 0.446 				   \\ \hline
		completa (sem ordenação) & 0.445   \\ \hline
		apenas pré-proc. & 0.505 		   \\ \hline
		apenas pós-proc. & 0.618 		   \\ \hline
		geral & 0.521 					   \\ \hline
		segmentação & 0.450 			   \\ \hline
	\end{tabular}
	\caption{Média de rácio de acerto das linhas da Partial GT.}
\end{table}



\begin{figure}[H]
	\centering
	\hspace*{-2cm}
	\includegraphics[width=1.2\textwidth]{images/resultados/graph_pgt_correct_order_ratio.png}
	\caption{Rácios de acerto da ordem das linhas da Partial GT das diferentes pipelines.}
	\label{fig:graph_pgt_correct_order_ratio}
\end{figure}


\begin{table}[H]
	\centering
	\begin{tabular}{|l|c|}
		\hline
		\textbf{Pipeline} & \textbf{Média} \\ \hline
		simples & 0.598 				   \\ \hline
		completa & 0.296 				   \\ \hline
		completa (sem ordenação) & 0.445   \\ \hline
		apenas pré-proc. & 0.5046 		   \\ \hline
		apenas pós-proc. & 0.322 		   \\ \hline
		geral & 0.351 					   \\ \hline
		segmentação & 0.294 			   \\ \hline
	\end{tabular}
	\caption{Média de rácio de ordem das linhas da Partial GT.}
\end{table}



No caso da GT parcial, maioritariamente dedicada à verificação de localização de algumas frases completas e da sua ordem no resultado final, observa-se novamente que na generalidade o OCR simples é superior. A primeira métrica é dependente da semelhança do texto com a GT e, como já supramencionado, na generalidade o OCR simples obteve também melhores resultados.  A segunda métrica é dependente da primeira, sendo portanto expectável esta correlação. 


\section{Conclusão}

Na generalidade, é possível concluir que, pelo menos dentro das configurações testadas, não houve nenhuma que excedesse os resultados de OCR simples, especialmente a aplicação total de técnicas de pré processamento (com uma configuração também para estas específica).

Em casos particulares, no entanto, podemos observar vários cenários de melhoria dos resultados.
Seguem-se exemplos:

\begin{itemize}
	\item Para o documento 'coimbra\_2' tem-se taxas semelhantes em termos de texto detetado pelas diferentes pipelines, no entanto, a taxa de acerto da partial GT para o OCR simples é o mais baixo, com a de pré processamento tendo total acerto, inclusive na ordenação da GT parcial.
	
	\item Para os documento 'a\_final\_8' e 'a\_final\_9' temos que o rácio de acerto da ordem das linhas da GT parcial não é de 1, sendo no máximo 0.5 para o primeiro. Porém, analisando os resultados localmente, temos que a ordem calculado para os dois é completamente correta para GT parcial (e quase totalmente para a GT), não sendo mais alta no gráfico apenas porque as linhas não foram todas detetadas devido a pequenos erros, como acentos ou pontuação errada. Ex.: "que usaste sempre dé tanta" (resultado) vs "que, usaste sempre de tanta" (GT)
	
\end{itemize}



Concluindo, os testes realizados permitem entender que a pipeline, no seu estado atual, tem o potencial para, quando devidamente configurada, melhorar os resultados relativamente ao uso base de OCR. Como esperava, também se verificou que a criação de uma configuração universal será difícil de alcançar. 

Deste modo, uma perspetiva de evolução passaria pela permissão de adaptação interna da pipeline, ou de uma aplicação semelhante do Toolkit, que procure adaptar de forma inteligente as suas configurações de acordo com o input. Além disso, contrastou-se o potencial da introdução de módulos de correção de texto, através da análise das GT parciais.




% aplicacao da pipeline nos casos de estudo
%% realcar situacoes particulares de sucesso ou de problemas


\section{OSDOCR Editor}


A última componente a analisar é o editor. Este tem como base o Toolkit e, como \textit{modus operandis}, a manipulação da estrutura OCR Tree. 

A sua proposta principal, como discutido nas proposições do capítulo \ref{cap_osdocr_filosofia}, é a disponibilização de um ambiente gráfico para fácil manipulação da estrutura de dados universal OCR Tree, consequentemente permitindo fazer reparos minuciosos nos resultados da pipeline e, servir como uma poderosa ferramenta de debugging da pipeline e do toolkit.

Através das suas funcionalidades, além da persistente dependência no Toolkit, também faz uso da pipeline - como é exemplo a aplicação de OCR localmente, ou reutilização do módulo de extração de output -, sendo portanto relevante como parcial resultado destes, especialmente do toolkit.

Como o toolkit, a avaliação desta componente não é direta, a não ser num cenário de extensa listagem das funcionalidades, isolados e combinadas, analisadas em diferentes casos de utilização. Alguns destes já foram descritos para funcionalidades mais relevantes e únicas desta componente, no seu capítulo de implementação. 

Outra forma de avaliar esta ferramenta, mais subjetiva embora não menos relevante, provém do uso de grupos de utilizadores. A estes poderiam ser dados conjuntos de tarefas a realizar, e registadas as dificuldades (ou falta delas) na sua execução, opiniões sobre as capacidades da ferramenta, e capacidade de as executar sem apoio externo. Esta metodologia não foi seguida, por restrições de tempo, mas é importante realçar a sua relevância no contexto de disponibilização da ferramenta para um público mais amplo, assim como a sua acessibilidade para sujeitos menos envolvidos no contexto técnico.

A avaliação seguida baseia-se então na apresentação das capacidades e adaptabilidade desta interface gráfica, através da listagem de casos de uso distintos e contrastantes.


\begin{itemize}\setlength\itemsep{-0.5em}
	\item Análise e manipulação de ficheiros do tipo OCR Tree, i.e. formato json e hocr : aplicabilidade para resultados de soluções além da OSDOCR pipeline.
	
	\item Visualização de diferentes níveis da OCR Tree.
	
	\item Filtragem da OCR Tree: através de filtros de tipo, texto, coordenadas e ID.
	
	\item Fácil retrocesso e reconstrução de operações complexas sob a OCR Tree: através de uma cache de OCR Tree.
	
	\item Limpeza de blocos de ruído de OCR Tree.
	
	\item Ajuste de dimensões e posicionamento de blocos da OCR Tree.
	
	\item Divisão de blocos através da ferramenta de corte : sem GUI seria necessário manualmente modificar as árvores, respetivamente atualizando o texto dos filhos e as suas coordenadas.
	
	\item Junção facilitada de blocos : sem GUI seria necessário modificar as árvores, mantendo apenas uma modificando as suas coordenadas e, especialmente, realocando os filhos de acordo com o seu posicionamento. Especialmente difícil na junção de árvores com texto que se intercala ou sobrepõe.
	
	\item (Re)categorização de blocos.
	
	\item Modificação direta do texto de blocos : manualmente seria necessário modificar as folhas para atualizar ou adicionar palavras, e criar novos nodos para cada linha e parágrafo.
	
	\item Criação de blocos não existentes.
	
	\item Realizar OCR num bloco : servindo para melhorar texto e a sua confiança por transcrição máquina.
	
	\item Realizar OCR num segmento da imagem: gera, de acordo com a segmentação do motor OCR, múltiplos blocos com uma só ferramenta.
	
	\item Aplicar pipeline localmente: permite adaptar o OCR realizado configurando a pipeline para lidar com o segmento da imagem escolhido. Ex.: pipeline com upscaling numa imagem inteira pode não detetar com tanta precisão texto de um título específico, comparado com a aplicação da mesma pipeline localmente (usualmente produto de dpi assumido pelo tesseract).
	
	\item (Re)ordenar blocos através da modificação do seu ID: modificação direta ou com ferramentas do Toolkit.
	
	
	\item Gerar OCR Tree numa imagem sem resultados.
	\item Visualizar resultados da OSDOCR Pipeline: útil para contexto de debugging e desenvolvimento.
	
	\item Corrigir resultados da OSDOCR Pipeline: sendo que este durante as vários módulos produz uma OCR Tree, pode-se corrigir um ponto específico.
	\item Visualizar resultados do OSDOCR Toolkit.
	
	\item Conversão de  OCR Tree em output textual simples.
	
	\item Conversão de OCR Tree em output textual de artigos.
	
	\item Segmentação da OCR Tree em artigos : manual ou utilizando métodos do toolkit.
	
	\item Manipulação de artigos : reordenação, atualização, inserção e remoção.
	
	\item Criação de OCR Tree para segmento de uma imagem com respetiva OCR Tree : com input de imagem e OCR Tree, a ferramenta de divisão de imagem gera uma nova imagem, partição da primeira, com uma OCR Tree constando cópias dos blocos que estavam inseridos e/ou intersetados (dependo das configurações do editor) na partição. Facilita assim a criação de resultados de OCR para partição de uma imagem que já possui resultados.
	
	\item Criação facilitada de resultados de confiança para jornais antigos manualmente: como intendido pela premissa da tese.
	
	\item Criação de resultados de confiança para documentos de outro tipo manualmente: não sendo tão focado para documentos que produzam resultados ruidosos, amplia a utilidade da solução; ex. documentos aplicáveis: banda desenhada, livros, revistas, recibos, etc..
	
	\item Suavização do processo de criação de ilustrações no contexto de OCR e manipulação da OCR Tree: como prova deste conceito, tem-se que a maioria das imagens desta dissertação fizeram uso do OSDOCR Editor
\end{itemize}


\chapter{Conclusões e trabalho futuro}
\label{cap_conclusao}

Neste capítulo será feito um sumário do trabalho e estudo realizado e uma introspeção sobre o trabalho futuro.

\section{Conclusões}

O projeto atual, propõe a concretização de uma ferramenta para melhorar os resultados de softwares de reconhecimento de caracteres em documentos estruturados antigos, em particular, jornais. Para isto, nesta primeira fase, foram definidos os objetivos principais do trabalho, assim como algumas vias de expansão consoante o desenrolar da sua implementação. Além disso, foi realizado um estudo sobre o estado da arte com base em dois aspetos principais: softwares \acrshort{ocr} e práticas comuns na sua utilização; e exploração sobre trabalhos relacionados a este tema ou técnicas relevantes para a proposta. Com isto, foi possível entender os desafios mais relevantes que se apresentam ao reconhecimento de caracteres, assim como os procedimentos \textit{standard} para os abordar, nomeadamente: pré processamento de imagem, pós processamento de texto, segmentação e métricas de validação; e algumas soluções focadas em tarefas similares ao do atual trabalho. Por último, realizou-se um compilado de algumas tarefas de implementação já realizadas que auxiliaram na perceção dos desafios impostos no tema e ao mesmo tempo uma melhor perceção sobre o funcionamento e capacidade da tecnologia \acrshort{ocr}.

\section{Perspetiva de trabalho futuro}

Partindo do estado atual do projeto, onde uma base de conhecimento do tema já foi concebida, os futuros passos seguirão maioritariamente na componente prática proposta, i.e. a construção das ferramentas para extração de conteúdo de jornais. Como mencionado nos objetivos, abre-se ainda a possibilidade para um aprofundamento na área de criação de léxicos entre versões de uma mesma linguagem para possibilitar a modernização do conteúdo extraído pela ferramenta principal. 
		

\renewcommand{\baselinestretch}{1}
\bibliographystyle{plainnat}
\bibliography{dissertation}
\printindex

\appendix
\renewcommand\chaptername{Apêndice}

%\chapter{Trabalho de apoio}
Resultados auxiliares.
\chapter{Detalhes dos resultados}


\begin{figure}[H]
	\centering
	\includegraphics[angle=90,width=0.3\textwidth]{images/resultados/graph_avg_text_conf.png}
	\caption{Valores de confiança média de texto das diferentes pipelines (alargado).}
	\label{fig:graph_avg_text_conf_large}
\end{figure}



\begin{figure}[H]
	\centering
	\includegraphics[angle=90,width=0.4\textwidth]{images/resultados/graph_gt_similiraty_cosine.png}
	\caption{Valores de similaridade do texto (similaridade por cosseno) com GT das diferentes pipelines (alargado).}
	\label{fig:graph_gt_similiraty_cosine_large}
\end{figure}


\begin{figure}[H]
	\centering
	\includegraphics[angle=90,width=0.4\textwidth]{images/resultados/graph_gt_word_hit_ratio.png}
	\caption{Rácios de aparição total de palavras da GT das diferentes pipelines (alargado).}
	\label{fig:graph_gt_word_hit_ratio_large}
\end{figure}


\begin{figure}[H]
	\centering
	\includegraphics[angle=90,width=0.4\textwidth]{images/resultados/graph_gt_unique_word_hit_ratio.png}
	\caption{Rácios de aparição de palavras distintas da GT das diferentes pipelines (alargado).}
	\label{fig:graph_gt_unique_word_hit_ratio_large}
\end{figure}


\begin{figure}[H]
	\centering
	\includegraphics[angle=90,width=0.4\textwidth]{images/resultados/graph_pgt_hit_ratio.png}
	\caption{Rácios de aparição de linhas da Partial GT das diferentes pipelines (alargado).}
	\label{fig:graph_pgt_hit_ratio_large}
\end{figure}


\begin{figure}[H]
	\centering
	\includegraphics[angle=90,width=0.4\textwidth]{images/resultados/graph_pgt_correct_order_ratio.png}
	\caption{Rácios de acerto da ordem das linhas da Partial GT das diferentes pipelines (alargado).}
	\label{fig:graph_pgt_correct_order_ratio_large}
\end{figure}



\begin{figure}[H]
	\centering
	\includegraphics[angle=90,width=0.4\textwidth]{images/resultados/graph_text_block_ratio.png}
	\caption{Rácios de número de blocos de texto relativo à pipeline simples (alargado).}
	\label{fig:graph_text_block_ratio_large}
\end{figure}

%\chapter{Listings}
Se for o caso.
\chapter{Ferramentas}
(Se for o caso)

Utilizadores de \Latex\ devem consultar \TUG,
o grupo de utilizadores \tug{\TeX}.

\pagestyle{empty}
\cleartoevenpage
\null
\thispagestyle{empty}
\pagecolor{PANTONECoolGray7C}
\afterpage{\nopagecolor}
\newpage

\begin{backcover}
\thispagestyle{empty}{~\vfill
\noindent
Coloque aqui informação sobre financiamento, projeto FCT, etc. em que o trabalho se enquadra. Deixe em branco caso contrário.
\vfill ~}
\end{backcover}



\end{document}
